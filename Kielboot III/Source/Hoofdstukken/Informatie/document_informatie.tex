\header{0}
\chapter*{Voorwoord}

\section{Voorwoord}
Na het afronden van de lesboeken voor Kielboot I en II, kon die voor Kielboot III natuurlijk niet uitblijven! Hoewel ik volledig onderschat had hoeveel meer werk dit boek zou zijn dan de vorige twee, is het ruim twee jaar na de eerste opzet toch gelukt dit boek af te ronden. Met ruim 90 extra figuren worden alle theorie-onderdelen van het KB III duidelijk en inzichtelijk behandeld.

- Christian Peppelman
\section{Dankwoord}
Na het maken van het eerste concept van dit lesboek heb ik van verschillende mensen om mij heen hele waardevolle feedback mogen ontvangen. Ik wil Wendela graag bedanken voor haar hulp met de taalkundige verbeteringen. Daarnaast wil ik ook de kielboot III cursisten van 2023 bedanken voor alle fouten en verbeteringen die zij hebben aangeleverd!

Ik wil ook graag de Katwijkse Zeeverkenners bedanken voor het online beschikbaar stellen van hun uitstekende lesboeken (\url{https://www.katwijksezeeverkenners.nl/cwo/instructieboeken/}). Het lesboek van de Katwijkse Zeeverkenners is een grote inspiratiebron geweest voor de figuren in dit lesboek.


\section{Lesstof verantwoording}
De lesstof die in dit boek aan bod komt, is gemaakt om zo goed mogelijk aan de eisen van de stichting Commissie Watersport Opleidingen (CWO) te voldoen voor de discipline kielboot III. Deze eisen zijn te vinden op \url{https://cwo.nl/leren-varen/kielboot}. Op sommige vlakken gaat dit boek uitgebreider in op de stof dan vanuit het CWO strikt noodzakelijk is. Hiervoor is gekozen omdat deze kennis een toegevoegde waarde kan bieden tijdens het zeilen op scouting.

\vfil\newpage

\section{Document Informatie}
\subsection*{Licentie}
\begin{figure}[H]
	\centering
	\begin{minipage}[t]{0.60\textwidth}
		\vspace{-1.80cm}
		Dit boek is uitgebracht onder een Creative Commons
		`Naamsvermelding-NietCommercieel-GelijkDelen 4.0 Internationaal' (CC BY-NC-SA 4.0) licentie. Voor meer informatie: \url{https://creativecommons.org/licenses/by-nc-sa/4.0/}
	\end{minipage}
	\hfill
	\begin{minipage}[b]{0.35\textwidth}
	\includegraphics[width=\textwidth]{../Hoofdstukken/Informatie/CC-BY-NC-SA.png}
\end{minipage}
\end{figure}
\subsection*{Auteur informatie}
Dit boek is geschreven door Christian Peppelman.\\ 
Voor contact, vragen of verbetering kun je mailen naar: \href{mailto:cwo@sintmaartengroep.nl}{CWO@sintmaartengroep.nl} 
\subsection*{Gebruik}
Om optimaal gebruik te kunnen maken van dit lesboek, deze graag laten drukken in een geniete brochure in kleur. Gelieve het boek niet thuis te printen, inscannen of vermenigvuldigen op een manier die negatieve invloed op de kwaliteit heeft. Voor de originele bestanden of gedrukte varianten kun je contact opnemen of kijken op \url{https://sintmaartengroep.nl/}
\subsection*{Thema}
Het thema waar dit boek op gebaseerd is heet `The Legrand Orange Book' en is ontworpen door Mathias Legrand. Het thema is gedownload op \url{https://nl.overleaf.com/latex/templates/} en valt onder een Creative Commons BY-NC-SA 3.0 licentie.
\subsection*{Versiebeheer}
\begin{table}[H]
	\centering
	\begin{tabular}{l|l|p{8cm}}
		\textbf{Versie} & \textbf{Datum} & \textbf{Omschrijving} \\ \hline
		0.1.0 & 22 maart 2020 & Eerste Opzet  \\ \hline
	    0.2.0 & 15 januari 2023 & Eerste Concept  \\ \hline
	    1.0.0 & 26 februari 2023 & Eerste Versie  \\
	\end{tabular}
\end{table}



\textit{Versie 1.0.0 \hspace{1 cm} 26 februari 2023 \hspace{1cm} Druk 1}
