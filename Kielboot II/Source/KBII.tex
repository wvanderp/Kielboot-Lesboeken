%%%%%%%%%%%%%%%%%%%%%%%%%%%%%%%%%%%%%%%%%
% The Legrand Orange Book
% LaTeX Template
% Version 2.3 (8/8/17)
%
% This template has been downloaded from:
% http://www.LaTeXTemplates.com
%
% Original author:
% Mathias Legrand (legrand.mathias@gmail.com) with modifications by:
% Vel (vel@latextemplates.com)
%
% License:
% CC BY-NC-SA 3.0 (http://creativecommons.org/licenses/by-nc-sa/3.0/)
%
% Compiling this template:
% This template uses biber for its bibliography and makeindex for its index.
% When you first open the template, compile it from the command line with the
% commands below to make sure your LaTeX distribution is configured correctly:
%
% 1) pdflatex main
% 2) makeindex main.idx -s StyleInd.ist
% 3) biber main
% 4) pdflatex main x 2
%
% After this, when you wish to update the bibliography/index use the appropriate
% command above and make sure to compile with pdflatex several times
% afterwards to propagate your changes to the document.
%
% This template also uses a number of packages which may need to be
% updated to the newest versions for the template to compile. It is strongly
% recommended you update your LaTeX distribution if you have any
% compilation errors.
%
% Important note:
% Chapter heading images should have a 2:1 width:height ratio,
% e.g. 920px width and 460px height.
%
%%%%%%%%%%%%%%%%%%%%%%%%%%%%%%%%%%%%%%%%%

%----------------------------------------------------------------------------------------
%	PACKAGES AND OTHER DOCUMENT CONFIGURATIONS
%----------------------------------------------------------------------------------------

\documentclass[11pt,fleqn]{book} % Default font size and left-justified equations
%----------------------------------------------------------------------------------------

%%%%%%%%%%%%%%%%%%%%%%%%%%%%%%%%%%%%%%%%%
% The Legrand Orange Book
% Structural Definitions File
% Version 2.0 (9/2/15)
%
% Original author:
% Mathias Legrand (legrand.mathias@gmail.com) with modifications by:
% Vel (vel@latextemplates.com)
%
% This file has been downloaded from:
% http://www.LaTeXTemplates.com
%
% License:
% CC BY-NC-SA 3.0 (http://creativecommons.org/licenses/by-nc-sa/3.0/)
%
%%%%%%%%%%%%%%%%%%%%%%%%%%%%%%%%%%%%%%%%%

%----------------------------------------------------------------------------------------
%	VARIOUS REQUIRED PACKAGES AND CONFIGURATIONS
%----------------------------------------------------------------------------------------
\usepackage[top=3cm,bottom=3cm,left=3cm,right=3cm,headsep=10pt,a4paper]{geometry} % Page margins
\usepackage[table]{xcolor}  % Required for specifying colors by name
\usepackage{graphicx} % Required for including pictures
\graphicspath{{Pictures/}} % Specifies the directory where pictures are stored

\usepackage{lipsum} % Inserts dummy text

\usepackage{wrapfig}

\usepackage{tikz} % Required for drawing custom shapes

\usepackage[dutch]{babel} % English language/hyphenation

\usepackage{enumitem} % Customize lists
\setlist{nolistsep} % Reduce spacing between bullet points and numbered lists

\usepackage{booktabs} % Required for nicer horizontal rules in tables

\usepackage{xcolor} % Required for specifying colors by name

\definecolor{ocre}{RGB}{0,161,255} % Define the orange color used for highlighting throughout the book
\definecolor{not}{RGB}{192,192,192} %


%----------------------------------------------------------------------------------------
%	FONTS
%----------------------------------------------------------------------------------------

\usepackage{avant} % Use the Avantgarde font for headings
%\usepackage{times} % Use the Times font for headings
\usepackage{mathptmx} % Use the Adobe Times Roman as the default text font together with math symbols from the Sym­bol, Chancery and Com­puter Modern fonts

\usepackage{microtype} % Slightly tweak font spacing for aesthetics
\usepackage[utf8]{inputenc} % Required for including letters with accents
\usepackage[T1]{fontenc} % Use 8-bit encoding that has 256 glyphs

%----------------------------------------------------------------------------------------
%	BIBLIOGRAPHY AND INDEX
%----------------------------------------------------------------------------------------

\usepackage[style=numeric,citestyle=numeric,sorting=none,sortcites=true,autopunct=true,babel=hyphen,hyperref=true,abbreviate=false,backref=true,backend=biber]{biblatex}
\addbibresource{bibliography.bib} % BibTeX bibliography file
\defbibheading{bibempty}{}

\usepackage{calc} % For simpler calculation - used for spacing the index letter headings correctly
\usepackage{makeidx} % Required to make an index
\makeindex % Tells LaTeX to create the files required for indexing

%----------------------------------------------------------------------------------------
%	MAIN TABLE OF CONTENTS
%----------------------------------------------------------------------------------------

\usepackage{titletoc} % Required for manipulating the table of contents

\contentsmargin{0cm} % Removes the default margin

% Part text styling
\titlecontents{part}[0cm]
{\addvspace{20pt}\centering\large\bfseries}
{}
{}
{}

% Chapter text styling
\titlecontents{chapter}[1.25cm] % Indentation
{\addvspace{12pt}\large\sffamily\bfseries} % Spacing and font options for chapters
{\color{ocre!60}\contentslabel[\Large\thecontentslabel]{1.25cm}\color{ocre}} % Chapter number
{\color{ocre}}
{\color{ocre!60}\normalsize\;\titlerule*[.5pc]{.}\;\thecontentspage} % Page number

% Section text styling
\titlecontents{section}[1.25cm] % Indentation
{\addvspace{3pt}\sffamily\bfseries} % Spacing and font options for sections
{\contentslabel[\thecontentslabel]{1.25cm}} % Section number
{}
{\hfill\color{black}\thecontentspage} % Page number
[]

% Subsection text styling
\titlecontents{subsection}[1.25cm] % Indentation
{\addvspace{1pt}\sffamily\small} % Spacing and font options for subsections
{\contentslabel[\thecontentslabel]{1.25cm}} % Subsection number
{}
{\ \titlerule*[.5pc]{.}\;\thecontentspage} % Page number
[]

% List of figures
\titlecontents{figure}[0em]
{\addvspace{-5pt}\sffamily}
{\thecontentslabel\hspace*{1em}}
{}
{\ \titlerule*[.5pc]{.}\;\thecontentspage}
[]

% List of tables
\titlecontents{table}[0em]
{\addvspace{-5pt}\sffamily}
{\thecontentslabel\hspace*{1em}}
{}
{\ \titlerule*[.5pc]{.}\;\thecontentspage}
[]

%----------------------------------------------------------------------------------------
%	MINI TABLE OF CONTENTS IN PART HEADS
%----------------------------------------------------------------------------------------

\titlecontents{lchapter}[0em] % Indenting
{\addvspace{15pt}\large\sffamily\bfseries} % Spacing and font options for chapters
{\color{ocre}\contentslabel[\Large\thecontentslabel]{1.25cm}\color{ocre}} % Chapter number
{}
{\color{ocre}\normalsize\sffamily\bfseries\;\titlerule*[.5pc]{.}\;\thecontentspage} % Page number

% Section text styling
\titlecontents{lsection}[0em] % Indenting
{\sffamily\small} % Spacing and font options for sections
{\contentslabel[\thecontentslabel]{1.25cm}} % Section number
{}
{}

% Subsection text styling
\titlecontents{lsubsection}[.5em] % Indentation
{\normalfont\footnotesize\sffamily} % Font settings
{}
{}
{}

%----------------------------------------------------------------------------------------
%	PAGE HEADERS
%----------------------------------------------------------------------------------------

\usepackage{fancyhdr} % Required for header and footer configuration

\pagestyle{fancy}
\renewcommand{\chaptermark}[1]{\markboth{\sffamily\normalsize\bfseries\chaptername\ \thechapter.\ #1}{}} % Chapter text font settings
\renewcommand{\sectionmark}[1]{}
%\renewcommand{\sectionmark}[1]{\markright{\sffamily\normalsize\thesection\hspace{5pt}#1}{}} % Section text font settings
%\fancyhf{} \fancyhead[LE,RO]{\sffamily\normalsize\thepage} %removed by CP
\fancyhf{} \fancyhead[LE,RO]{\sffamily\normalsize} % Font setting for the page number in the header
\fancyhead[LO]{\rightmark} % Print the nearest section name on the left side of odd pages
\fancyhead[RE]{\leftmark} % Print the current chapter name on the right side of even pages
\fancyfoot[LE,RO]{\sffamily\normalsize\thepage} %added by CP
%\fancyfoot[C]{CONCEPT VERSIE}
\renewcommand{\headrulewidth}{0.5pt} % Width of the rule under the header
\addtolength{\headheight}{2.5pt} % Increase the spacing around the header slightly
\renewcommand{\footrulewidth}{0pt} % Removes the rule in the footer
\fancypagestyle{plain}{\fancyhead{}\renewcommand{\headrulewidth}{0pt}} % Style for when a plain pagestyle is specified

% Removes the header from odd empty pages at the end of chapters
\makeatletter
\renewcommand{\cleardoublepage}{
\clearpage\ifodd\c@page\else
\hbox{}
\vspace*{\fill}
\thispagestyle{empty}
\newpage
\fi}

%----------------------------------------------------------------------------------------
%	THEOREM STYLES
%----------------------------------------------------------------------------------------

\usepackage{amsmath,amsfonts,amssymb,amsthm} % For math equations, theorems, symbols, etc

\newcommand{\intoo}[2]{\mathopen{]}#1\,;#2\mathclose{[}}
\newcommand{\ud}{\mathop{\mathrm{{}d}}\mathopen{}}
\newcommand{\intff}[2]{\mathopen{[}#1\,;#2\mathclose{]}}
\newtheorem{notation}{Notation}[chapter]

% Boxed/framed environments
\newtheoremstyle{ocrenumbox}% % Theorem style name
{0pt}% Space above
{0pt}% Space below
{\normalfont}% % Body font
{}% Indent amount
{\small\bf\sffamily\color{ocre}}% % Theorem head font
{\;}% Punctuation after theorem head
{0.25em}% Space after theorem head
{\small\sffamily\color{ocre}\thmname{#1}\nobreakspace\thmnumber{\@ifnotempty{#1}{}\@upn{#2}}% Theorem text (e.g. Theorem 2.1)
\thmnote{\nobreakspace\the\thm@notefont\sffamily\bfseries\color{black}---\nobreakspace#3.}} % Optional theorem note
\renewcommand{\qedsymbol}{$\blacksquare$}% Optional qed square

\newtheoremstyle{blacknumex}% Theorem style name
{5pt}% Space above
{5pt}% Space below
{\normalfont}% Body font
{} % Indent amount
{\small\bf\sffamily}% Theorem head font
{\;}% Punctuation after theorem head
{0.25em}% Space after theorem head
{\small\sffamily{\tiny\ensuremath{\blacksquare}}\nobreakspace\thmname{#1}\nobreakspace\thmnumber{\@ifnotempty{#1}{}\@upn{#2}}% Theorem text (e.g. Theorem 2.1)
\thmnote{\nobreakspace\the\thm@notefont\sffamily\bfseries---\nobreakspace#3.}}% Optional theorem note

\newtheoremstyle{blacknumbox} % Theorem style name
{0pt}% Space above
{0pt}% Space below
{\normalfont}% Body font
{}% Indent amount
{\small\bf\sffamily}% Theorem head font
{\;}% Punctuation after theorem head
{0.25em}% Space after theorem head
{\small\sffamily\thmname{#1}\nobreakspace\thmnumber{\@ifnotempty{#1}{}\@upn{#2}}% Theorem text (e.g. Theorem 2.1)
\thmnote{\nobreakspace\the\thm@notefont\sffamily\bfseries---\nobreakspace#3.}}% Optional theorem note

% Non-boxed/non-framed environments
\newtheoremstyle{ocrenum}% % Theorem style name
{5pt}% Space above
{5pt}% Space below
{\normalfont}% % Body font
{}% Indent amount
{\small\bf\sffamily\color{ocre}}% % Theorem head font
{\;}% Punctuation after theorem head
{0.25em}% Space after theorem head
{\small\sffamily\color{ocre}\thmname{#1}\nobreakspace\thmnumber{\@ifnotempty{#1}{}\@upn{#2}}% Theorem text (e.g. Theorem 2.1)
\thmnote{\nobreakspace\the\thm@notefont\sffamily\bfseries\color{black}---\nobreakspace#3.}} % Optional theorem note
\renewcommand{\qedsymbol}{$\blacksquare$}% Optional qed square
\makeatother

% Defines the theorem text style for each type of theorem to one of the three styles above
\newcounter{dummy}
\numberwithin{dummy}{section}
\theoremstyle{ocrenumbox}
\newtheorem{theoremeT}[dummy]{Theorem}
\newtheorem{problem}{Problem}[chapter]
\newtheorem{exerciseT}{Exercise}[chapter]
\theoremstyle{blacknumex}
\newtheorem{exampleT}{Example}[chapter]
\theoremstyle{blacknumbox}
\newtheorem{vocabulary}{Vocabulary}[chapter]
\newtheorem{definitionT}{Definition}[section]
\newtheorem{corollaryT}[dummy]{Corollary}
\theoremstyle{ocrenum}
\newtheorem{proposition}[dummy]{Proposition}

%----------------------------------------------------------------------------------------
%	DEFINITION OF COLORED BOXES
%----------------------------------------------------------------------------------------

\RequirePackage[framemethod=default]{mdframed} % Required for creating the theorem, definition, exercise and corollary boxes

% Theorem box
\newmdenv[skipabove=7pt,
skipbelow=7pt,
backgroundcolor=black!5,
linecolor=ocre,
innerleftmargin=5pt,
innerrightmargin=5pt,
innertopmargin=5pt,
leftmargin=0cm,
rightmargin=0cm,
innerbottommargin=5pt]{tBox}

% Exercise box
\newmdenv[skipabove=7pt,
skipbelow=7pt,
rightline=false,
leftline=true,
topline=false,
bottomline=false,
backgroundcolor=ocre!10,
linecolor=ocre,
innerleftmargin=5pt,
innerrightmargin=5pt,
innertopmargin=5pt,
innerbottommargin=5pt,
leftmargin=0cm,
rightmargin=0cm,
linewidth=4pt]{eBox}	

% Definition box
\newmdenv[skipabove=7pt,
skipbelow=7pt,
rightline=false,
leftline=true,
topline=false,
bottomline=false,
linecolor=ocre,
innerleftmargin=5pt,
innerrightmargin=5pt,
innertopmargin=0pt,
leftmargin=0cm,
rightmargin=0cm,
linewidth=4pt,
innerbottommargin=0pt]{dBox}	

% Corollary box
\newmdenv[skipabove=7pt,
skipbelow=7pt,
rightline=false,
leftline=true,
topline=false,
bottomline=false,
linecolor=gray,
backgroundcolor=black!5,
innerleftmargin=5pt,
innerrightmargin=5pt,
innertopmargin=5pt,
leftmargin=0cm,
rightmargin=0cm,
linewidth=4pt,
innerbottommargin=5pt]{cBox}

% Creates an environment for each type of theorem and assigns it a theorem text style from the "Theorem Styles" section above and a colored box from above
\newenvironment{theorem}{\begin{tBox}\begin{theoremeT}}{\end{theoremeT}\end{tBox}}
\newenvironment{exercise}{\begin{eBox}\begin{exerciseT}}{\hfill{\color{ocre}\tiny\ensuremath{\blacksquare}}\end{exerciseT}\end{eBox}}
\newenvironment{definition}{\begin{dBox}\begin{definitionT}}{\end{definitionT}\end{dBox}}	
\newenvironment{example}{\begin{exampleT}}{\hfill{\tiny\ensuremath{\blacksquare}}\end{exampleT}}		
\newenvironment{corollary}{\begin{cBox}\begin{corollaryT}}{\end{corollaryT}\end{cBox}}	

%----------------------------------------------------------------------------------------
%	REMARK ENVIRONMENT
%----------------------------------------------------------------------------------------

\newenvironment{remark}{\par\vspace{10pt}\small % Vertical white space above the remark and smaller font size
\begin{list}{}{
\leftmargin=35pt % Indentation on the left
\rightmargin=25pt}\item\ignorespaces % Indentation on the right
\makebox[-2.5pt]{\begin{tikzpicture}[overlay]
\node[draw=ocre!60,line width=1pt,circle,fill=ocre!25,font=\sffamily\bfseries,inner sep=2pt,outer sep=0pt] at (-15pt,0pt){\textcolor{ocre}{R}};\end{tikzpicture}} % Orange R in a circle
\advance\baselineskip -1pt}{\end{list}\vskip5pt} % Tighter line spacing and white space after remark

%----------------------------------------------------------------------------------------
%	SECTION NUMBERING IN THE MARGIN
%----------------------------------------------------------------------------------------

\makeatletter
\renewcommand{\@seccntformat}[1]{\llap{\textcolor{ocre}{\csname the#1\endcsname}\hspace{1em}}}
\renewcommand{\section}{\@startsection{section}{1}{\z@}
{-4ex \@plus -1ex \@minus -.4ex}
{1ex \@plus.2ex }
{\normalfont\large\sffamily\bfseries}}
\renewcommand{\subsection}{\@startsection {subsection}{2}{\z@}
{-3ex \@plus -0.1ex \@minus -.4ex}
{0.5ex \@plus.2ex }
{\normalfont\sffamily\bfseries}}
\renewcommand{\subsubsection}{\@startsection {subsubsection}{3}{\z@}
{-2ex \@plus -0.1ex \@minus -.2ex}
{.2ex \@plus.2ex }
{\normalfont\small\sffamily\bfseries}}
\renewcommand\paragraph{\@startsection{paragraph}{4}{\z@}
{-2ex \@plus-.2ex \@minus .2ex}
{.1ex}
{\normalfont\small\sffamily\bfseries}}

%----------------------------------------------------------------------------------------
%	PART HEADINGS
%----------------------------------------------------------------------------------------

% numbered part in the table of contents
\newcommand{\@mypartnumtocformat}[2]{%
\setlength\fboxsep{0pt}%
\noindent\colorbox{ocre!20}{\strut\parbox[c][.7cm]{\ecart}{\color{ocre!70}\Large\sffamily\bfseries\centering#1}}\hskip\esp\colorbox{ocre!40}{\strut\parbox[c][.7cm]{\linewidth-\ecart-\esp}{\Large\sffamily\centering#2}}}%
%%%%%%%%%%%%%%%%%%%%%%%%%%%%%%%%%%
% unnumbered part in the table of contents
\newcommand{\@myparttocformat}[1]{%
\setlength\fboxsep{0pt}%
\noindent\colorbox{ocre!40}{\strut\parbox[c][.7cm]{\linewidth}{\Large\sffamily\centering#1}}}%
%%%%%%%%%%%%%%%%%%%%%%%%%%%%%%%%%%
\newlength\esp
\setlength\esp{4pt}
\newlength\ecart
\setlength\ecart{1.2cm-\esp}
\newcommand{\thepartimage}{}%
\newcommand{\partimage}[1]{\renewcommand{\thepartimage}{#1}}%
\def\@part[#1]#2{%
\ifnum \c@secnumdepth >-2\relax%
\refstepcounter{part}%
\addcontentsline{toc}{part}{\texorpdfstring{\protect\@mypartnumtocformat{\thepart}{#1}}{\partname~\thepart\ ---\ #1}}
\else%
\addcontentsline{toc}{part}{\texorpdfstring{\protect\@myparttocformat{#1}}{#1}}%
\fi%
\startcontents%
\markboth{}{}%
{\thispagestyle{empty}%
\begin{tikzpicture}[remember picture,overlay]%
\node at (current page.north west){\begin{tikzpicture}[remember picture,overlay]%	
\fill[ocre!20](0cm, 0cm) rectangle (\paperwidth,-\paperheight);
\node[anchor=north] at (4cm,-3.25cm){\color{ocre!40}\fontsize{220}{100}\sffamily\bfseries\thepart};

\ifnum \value{part}=1 %added by CP. If first chapter
\node[anchor=south east] at (\paperwidth-1cm,-\paperheight+2.5cm){\parbox[t][][t]{8.5cm}{
		\printcontents{l}{0}{\setcounter{tocdepth}{1}}%
}};

\makeatletter
\newcommand \Dotfill {\leavevmode \cleaders \hb@xt@ .22cm{\hss .\hss }\hfill \kern \z@}
\makeatother
\node[anchor=south east] at (\paperwidth-1cm,-\paperheight+1cm) {\parbox[t][][t]{10cm}{
		\large\sffamily\bfseries\large \color{ocre}
		\begin{tabular}{p{0.9cm} p{8.5cm}}
		\Large II & \hyperref[part:oefen]{Oefenvragen} \normalsize\Dotfill \vspace{0.05cm} \hspace{0.05cm} \pageref{part:oefen} \\
		\Large III & \hyperref[part:manoeuvre]{Zeilmanoeuvres} \normalsize\Dotfill \hspace{0.05cm} \pageref{part:manoeuvre}
		\end{tabular}
}};
\else
\node[anchor=south east] at (\paperwidth-1cm,-\paperheight+1cm){\parbox[t][][t]{8.5cm}{
		\printcontents{l}{0}{\setcounter{tocdepth}{1}}%
}};
\fi %added by CP. If first chapter



\node[anchor=north east] at (\paperwidth-1.5cm,-3.25cm){\parbox[t][][t]{15cm}{\strut\raggedleft\color{white}\fontsize{30}{30}\sffamily\bfseries#2}};
\end{tikzpicture}};
\end{tikzpicture}}%
\@endpart}
\def\@spart#1{%
\startcontents%
\phantomsection
{\thispagestyle{empty}%
\begin{tikzpicture}[remember picture,overlay]%
\node at (current page.north west){\begin{tikzpicture}[remember picture,overlay]%	
\fill[ocre!20](0cm,0cm) rectangle (\paperwidth,-\paperheight);
\node[anchor=north east] at (\paperwidth-1.5cm,-3.25cm){\parbox[t][][t]{15cm}{\strut\raggedleft\color{white}\fontsize{30}{30}\sffamily\bfseries#1}};
\end{tikzpicture}};
\end{tikzpicture}}
\addcontentsline{toc}{part}{\texorpdfstring{%
\setlength\fboxsep{0pt}%
\noindent\protect\colorbox{ocre!40}{\strut\protect\parbox[c][.7cm]{\linewidth}{\Large\sffamily\protect\centering #1\quad\mbox{}}}}{#1}}%
\@endpart}
\def\@endpart{\vfil\newpage
\if@twoside
\if@openright
\null
\thispagestyle{empty}%
\newpage
\fi
\fi
\if@tempswa
\twocolumn
\fi}

%----------------------------------------------------------------------------------------
%	CHAPTER HEADINGS
%----------------------------------------------------------------------------------------

% A switch to conditionally include a picture, implemented by Christian Hupfer
\newif\ifusechapterimage
\usechapterimagetrue
\newcommand{\thechapterimage}{}%
\newcommand{\chapterimage}[1]{\ifusechapterimage\renewcommand{\thechapterimage}{#1}\fi}%
\newcommand{\autodot}{.}
\def\@makechapterhead#1{%
{\parindent \z@ \raggedright \normalfont
\ifnum \c@secnumdepth >\m@ne
\if@mainmatter
\begin{tikzpicture}[remember picture,overlay]
\node at (current page.north west)
{\begin{tikzpicture}[remember picture,overlay]
\node[anchor=north west,inner sep=0pt] at (0,0) {\ifusechapterimage\includegraphics[width=\paperwidth]{\thechapterimage}\fi};
%\textbf{\draw[anchor=west] (0cm,-4.6cm) node [fill=black,fill opacity=0.4,inner sep=12pt]{\strut\makebox[22cm]{}};}

%added by christian
{\ifusechapterimage
\textbf{\draw[anchor=west] (0cm,-4.6cm) node [fill=black,fill opacity=0.4,inner sep=12pt]{\strut\makebox[22cm]{}};}
\draw[anchor=west] (1.5cm,-4.7cm) node {\Huge\sffamily\bfseries\color{white}\thechapter\autodot~#1\strut};
\else
\textbf{\draw[anchor=west] (0cm,-1cm) node [fill=black,fill opacity=0.4,inner sep=12pt]{\strut\makebox[22cm]{}};}
\draw[anchor=west] (1.5cm,-1.1cm) node {\Huge\sffamily\bfseries\color{white}\thechapter\autodot~#1\strut};
\fi};

%\draw[anchor=west] (1.5cm,-4.7cm) node {\Huge\sffamily\bfseries\color{white}\thechapter\autodot~#1\strut};
\end{tikzpicture}};
\end{tikzpicture}
\else
\begin{tikzpicture}[remember picture,overlay]
\node at (current page.north west)
{\begin{tikzpicture}[remember picture,overlay]
\node[anchor=north west,inner sep=0pt] at (0,0) {\ifusechapterimage\includegraphics[width=\paperwidth]{\thechapterimage}\fi};
\draw[anchor=west] (10.1cm,-10.02cm) node {\Huge\sffamily\bfseries\color{white}#1\strut};
\end{tikzpicture}};
\end{tikzpicture}
\fi\fi\par\vspace*{100\p@}}}

%-------------------------------------------

\def\@makeschapterhead#1{%
\begin{tikzpicture}[remember picture,overlay]
\node at (current page.north west)
{\begin{tikzpicture}[remember picture,overlay]
\node[anchor=north west,inner sep=0pt] at (0,0) {\ifusechapterimage\includegraphics[width=\paperwidth]{\thechapterimage}\fi};
\draw[anchor=west] (13cm,-10.02cm) node {\Huge\sffamily\bfseries\color{white}#1\strut};
\end{tikzpicture}};
\end{tikzpicture}
\par\vspace*{230\p@}}
\makeatother

%----------------------------------------------------------------------------------------
%	HYPERLINKS IN THE DOCUMENTS
%----------------------------------------------------------------------------------------

\usepackage{hyperref}
\hypersetup{hidelinks,backref=true,pagebackref=true,hyperindex=true,colorlinks=false,breaklinks=true,urlcolor= ocre,bookmarks=true,bookmarksopen=false,
	pdftitle={Kielboot II - Scouting Sint Maarten},
	pdfauthor={Christian Peppelman},
	pdfsubject={Kielboot II - Zeiltheorie met focus op Scouting},
	pdfkeywords={CWO Kielboot II},
}
\usepackage{bookmark}
\bookmarksetup{
open,
numbered,
addtohook={%
\ifnum\bookmarkget{level}=0 % chapter
\bookmarksetup{bold}%
\fi
\ifnum\bookmarkget{level}=-1 % part
\bookmarksetup{color=ocre,bold}%
\fi
}
}

%----------------------------------------------------------------------------------------
%	Excercises
%----------------------------------------------------------------------------------------
\newcommand{\question}[2]{{\sffamily\bfseries{\textcolor{ocre}{\hspace{-0.5cm}#1. }{#2}}}}

\newcommand{\answerTextFour}[4]{
\begin{enumerate}[topsep=-2pt, label=\Alph*.]
    \item #1
    \item #2
    \item #3
    \item #4
\end{enumerate}}


\newcommand{\answerTextPicture}[5]{
\begin{figure}[H]	
	\vspace{-10px}
     \begin{minipage}[]{0.70\textwidth}
        \begin{enumerate}[topsep=0pt, label=\Alph*.]
            \item #1
            \item #2
            \item #3
            \item #4
        \end{enumerate}
      \end{minipage}
      \begin{minipage}[]{0.29\textwidth}
            \begin{figure}[H]
            \includegraphics[width=0.80\textwidth,right]{#5}
            \end{figure}
      \end{minipage}
  \vspace{-10px}
\end{figure}
}

\newcommand{\answerPicture}[4]{
\begin{figure}[H]
  \centering
  \begin{minipage}[b]{0.23\textwidth}
    \includegraphics[width=\textwidth]{#1}
    \centering
    A
  \end{minipage}
  \hfill
  \begin{minipage}[b]{0.23\textwidth}
    \includegraphics[width=\textwidth]{#2}
    \centering
    B
  \end{minipage}
  \hfill
  \begin{minipage}[b]{0.23\textwidth}
    \includegraphics[width=\textwidth]{#3}
    \centering
    C
  \end{minipage}
  \hfill
  \begin{minipage}[b]{0.23\textwidth}
    \includegraphics[width=\textwidth]{#4}
    \centering
    D
  \end{minipage}
\end{figure}
}
 % Insert the commands.tex file which contains the majority of the structure behind the template
\usepackage{graphicx}
\usepackage{lscape}
\usepackage{pgfgantt}
\usepackage{multicol}
\usepackage{caption}
\usepackage{float}
\usepackage{pdfpages}
\usepackage{wrapfig}
\usepackage{enumitem}
\usepackage[export]{adjustbox}
\captionsetup[table]{skip=5pt}
\usepackage{parskip} %CP
\usepackage{geometry} %for margin notes
\usepackage{marginnote} %for margin notes
\usepackage{array}
\usepackage{multirow}

\pdfinclusioncopyfonts=1 % This fixes missing characher form figures
\pdfsuppresswarningpagegroup=1 % ignore page group warning

%Header and front folder selection

%%% Sint Maarten %%%
\newcommand{\header}[1]{\chapterimage{Banners/header_#1.png}}
\newcommand{\omslag}{Omslag/omslag.pdf}

\begin{document}

%----------------------------------------------------------------------------------------
%	TITLE PAGE
%----------------------------------------------------------------------------------------

\begingroup
\thispagestyle{empty}
\begin{tikzpicture}[remember picture,overlay]
\node[inner sep=0pt] (background) at (current page.center) {\includegraphics[width=\paperwidth, page = 1]{\omslag}};
\end{tikzpicture}
\vfill
\endgroup

%----------------------------------------------------------------------------------------
%	COPYRIGHT PAGE
%----------------------------------------------------------------------------------------

\newpage
{\sffamily\bfseries Naam: }\rule{60mm}{.1pt}%
~\vfill
\thispagestyle{empty}
Voorpagina ontworpen door Christian Peppelman 2021. Achtergrond foto: Visit Aalsmeer. Website: \url{https://www.visitaalsmeer.nl/10x-weetjes-de-westeinderplassen/}. Tekening lelievlet gebaseerd op \textit{CWO Instructieboek} van de Katwijkse Zeeverkenners (zie dankwoord).
Handelsmerken op het voorblad zijn van respectievelijke stichtingen of organisaties. De CWO en Scouting Nederland zijn niet betrokken geweest bij het opstellen van dit lesboek.

\textit{Opgeleverd op \today} % Printing/edition date


\header{0}
\chapter*{Voorwoord}

\section{Voorwoord}
Na het afronden van de lesboeken voor Kielboot I en II, kon die voor Kielboot III natuurlijk niet uitblijven! Hoewel ik volledig onderschat had hoeveel meer werk dit boek zou zijn dan de vorige twee, is het ruim twee jaar na de eerste opzet toch gelukt dit boek af te ronden. Met ruim 90 extra figuren worden alle theorie-onderdelen van het KB III duidelijk en inzichtelijk behandeld.

- Christian Peppelman
\section{Dankwoord}
Na het maken van het eerste concept van dit lesboek heb ik van verschillende mensen om mij heen hele waardevolle feedback mogen ontvangen. Ik wil Wendela graag bedanken voor haar hulp met de taalkundige verbeteringen. Daarnaast wil ik ook de kielboot III cursisten van 2023 bedanken voor alle fouten en verbeteringen die zij hebben aangeleverd!

Ik wil ook graag de Katwijkse Zeeverkenners bedanken voor het online beschikbaar stellen van hun uitstekende lesboeken (\url{https://www.katwijksezeeverkenners.nl/cwo/instructieboeken/}). Het lesboek van de Katwijkse Zeeverkenners is een grote inspiratiebron geweest voor de figuren in dit lesboek.


\section{Lesstof verantwoording}
De lesstof die in dit boek aan bod komt, is gemaakt om zo goed mogelijk aan de eisen van de stichting Commissie Watersport Opleidingen (CWO) te voldoen voor de discipline kielboot III. Deze eisen zijn te vinden op \url{https://cwo.nl/leren-varen/kielboot}. Op sommige vlakken gaat dit boek uitgebreider in op de stof dan vanuit het CWO strikt noodzakelijk is. Hiervoor is gekozen omdat deze kennis een toegevoegde waarde kan bieden tijdens het zeilen op scouting.

\vfil\newpage

\section{Document Informatie}
\subsection*{Licentie}
\begin{figure}[H]
	\centering
	\begin{minipage}[t]{0.60\textwidth}
		\vspace{-1.80cm}
		Dit boek is uitgebracht onder een Creative Commons
		`Naamsvermelding-NietCommercieel-GelijkDelen 4.0 Internationaal' (CC BY-NC-SA 4.0) licentie. Voor meer informatie: \url{https://creativecommons.org/licenses/by-nc-sa/4.0/}
	\end{minipage}
	\hfill
	\begin{minipage}[b]{0.35\textwidth}
	\includegraphics[width=\textwidth]{../Hoofdstukken/Informatie/CC-BY-NC-SA.png}
\end{minipage}
\end{figure}
\subsection*{Auteur informatie}
Dit boek is geschreven door Christian Peppelman.\\ 
Voor contact, vragen of verbetering kun je mailen naar: \href{mailto:cwo@sintmaartengroep.nl}{CWO@sintmaartengroep.nl} 
\subsection*{Gebruik}
Om optimaal gebruik te kunnen maken van dit lesboek, deze graag laten drukken in een geniete brochure in kleur. Gelieve het boek niet thuis te printen, inscannen of vermenigvuldigen op een manier die negatieve invloed op de kwaliteit heeft. Voor de originele bestanden of gedrukte varianten kun je contact opnemen of kijken op \url{https://sintmaartengroep.nl/}
\subsection*{Thema}
Het thema waar dit boek op gebaseerd is heet `The Legrand Orange Book' en is ontworpen door Mathias Legrand. Het thema is gedownload op \url{https://nl.overleaf.com/latex/templates/} en valt onder een Creative Commons BY-NC-SA 3.0 licentie.
\subsection*{Versiebeheer}
\begin{table}[H]
	\centering
	\begin{tabular}{l|l|p{8cm}}
		\textbf{Versie} & \textbf{Datum} & \textbf{Omschrijving} \\ \hline
		0.1.0 & 22 maart 2020 & Eerste Opzet  \\ \hline
	    0.2.0 & 15 januari 2023 & Eerste Concept  \\ \hline
	    1.0.0 & 26 februari 2023 & Eerste Versie  \\
	\end{tabular}
\end{table}



\textit{Versie 1.0.0 \hspace{1 cm} 26 februari 2023 \hspace{1cm} Druk 1}


%----------------------------------------------------------------------------------------
%	TABLE OF CONTENTS
%----------------------------------------------------------------------------------------

%\usechapterimagefalse % If you don't want to include a chapter image, use this to toggle images off - it can be enabled later with \usechapterimagetrue

\pagestyle{empty} % No headers

%\tableofcontents % Print the table of contents itself

%\cleardoublepage % Forces the first chapter to start on an odd page so it's on the right

\pagestyle{fancy} % Print headers again
\part{Theorie Lessen}
\header{2}
\chapter{Bootonderdelen \& Zeiltermen}
\section{Inleiding}
In dit hoofdstuk komen de verschillende onderdelen van de boot en een aantal zeiltermen aan bod. Deze termen en onderdelen zijn belangrijk om de volgende hoofdstukken goed te begrijpen.

\section{Zeiltermen}
Voor duidelijke communicatie tijdens de les en in de boot is het van belang dat je een aantal zeiltermen kent. De belangrijkste termen worden hieronder besproken.

\subsection{Bakboord, Stuurboord, Loef en Lij}
Bakboord en stuurboord zijn het links en rechts \textbf{van de boot}, gezien vanaf het achterdek. Je moet altijd met de vaarrichting mee kijken. Loef en lij zeggen iets over de wind ten opzichte van je boot. De kant waar de wind de boot in komt, is de loefzijde, ook wel de hoge kant genoemd. De kant waar de wind de boot verlaat heet de lijzijde of lage kant. De hogerwal is de wal waar de wind vandaan komt. De lagerwal is de wal waar de wind naartoe waait. Al deze termen zijn te zien in figuur \ref{pic:hoog_laag}. 
\begin{figure}[ht]
	\centering
	\includegraphics[width=0.8\textwidth]{Hoofdstukken/Onderdelen/pdf/wallen.pdf}
	\caption{Hoger- en Lagerwal}
	\centering
	\label{pic:hoog_laag}
\end{figure}

\subsection{Boven- en benedenwinds}
\begin{figure}[H]
	\centering
	\begin{minipage}[t]{0.55\textwidth}
		\vspace{-4cm}
		Op het water kan je vaak op twee manieren ergens langs varen: bovenwinds en benedenwinds. Bovenwinds houdt in dat je ergens langs vaart aan de kant waar de wind ernaartoe blaast, de hoge kant van het object. Benedenwinds is het tegenovergestelde: dit is de kant waar de wind van het object weg blaast en dus de lage kant van het object. Deze termen zijn te zien in figuur \ref{pic:boven_benedenwinds}.
	\end{minipage}
	\hfill
	\begin{minipage}[b]{0.40\textwidth}
		\centering
		\includegraphics[width=\textwidth]{Hoofdstukken/Onderdelen/pdf/boven_en_benedenwinds.pdf}
		\caption{}
		\centering
		\label{pic:boven_benedenwinds}
	\end{minipage}
\end{figure}

\vfil\newpage

\subsection{Koersen}
Een koers vertelt iets over hoe je boot ligt ten opzichte van de wind. Alle koersen kun je zowel over bakboord, als stuurboord varen, behalve in de wind. Een overzicht van de koersen is te zien in figuur \ref{pic:koersen}. Wanneer je van koers verandert en naar de wind toe draait, loef je op. Wanneer je van de wind wegdraait heet dit afvallen. 

\begin{figure}[h]
	\centering
	\includegraphics[width=0.9\textwidth]{Hoofdstukken/Onderdelen/pdf/koersen.pdf}
	\caption{Windkoersen}
	\label{pic:koersen}
\end{figure}


\subsection{Overstag en opkruisen}
Wanneer je overstag gaat, ga je van aan de wind over de ene boeg naar aan de wind over de andere boeg. Bijvoorbeeld van aan de wind over bakboord naar aan de wind over stuurboord. Door meerdere malen achter elkaar overstag te gaan kun je naar een punt zeilen dat tegen de wind in ligt. Dit heet opkruisen of laveren en is te zien in figuur \ref{pic:opkruisen}. Het stuk dat je aan de wind vaart de tussen twee overstagen noem je een slag.  

In figuur \ref{pic:kort_lang} ligt het vaarwater niet exact in de wind, maar komt de wind met een kleine hoek naar binnen. Hierdoor zijn de slagen niet meer van gelijke lengte. Er is dan sprake van een korte slag en een lange slag.

\begin{figure}[h]
	\centering
	\begin{minipage}{0.40\textwidth}
		\centering
		\includegraphics[width=0.6\textwidth]{Hoofdstukken/Onderdelen/pdf/opkruisen.pdf}
		\caption{Opkruisen}
		\centering
		\label{pic:opkruisen}
	\end{minipage}
	\begin{minipage}{0.40\textwidth}
		\centering
		\includegraphics[width=0.6\textwidth]{Hoofdstukken/Onderdelen/pdf/korte_lange_Slag.pdf}
		\caption{Korte en Lange Slag}
		\label{pic:kort_lang}
	\end{minipage}
\end{figure}

\newpage

Door gebruik te maken van deze verschillende slaglengtes kun je zo effectief mogelijk opkruisen. In de korte slag probeer je zo veel mogelijk snelheid te maken. Vaar hier voor desnoods wat lager aan de wind om extra snelheid te maken. Deze snelheid gebruik je vervolgens in je lange slag. In deze slag probeer je zo veel mogelijk hoogte te winnen. Door scherp aan de wind te varen en de gewonnen snelheid uit de korte slag te gebruiken, leg je de meeste meters af.

\subsection{Gijpen en `binnen de wind varen'}
\begin{figure}[H]
	\centering
	\begin{minipage}[t]{0.78\textwidth}
		\vspace{-5cm}
		Gijpen kun je ook wel beschouwen als het tegenovergestelde van een overstag. Je gaat hier namelijk van voor de wind over de ene boeg naar voor de wind over de andere boeg. Bijvoorbeeld van voor de wind over bakboord naar voor de wind over stuurboord. Als je de gijp en de overstag in de windroos in figuur \ref{pic:koersen} plaatst zie je dat ze precies tegenover elkaar liggen.\\
		
		
		Door vlak voor het gijpen iets extra af te vallen, gijpt je zeil eenvoudiger. Dit extra afvallen is te zien in figuur \ref{pic:binnen_wind}. De onderste boot vaart exact voor de wind. De bovenste boot vaart richting ruime wind over stuurboord, maar met zijn zeil nog aan bakboord. Dit heet `binnen de wind' varen.
	\end{minipage}
	\hfill
	\begin{minipage}[b]{0.18\textwidth}
		\centering
		\includegraphics[width=0.7\textwidth]{Hoofdstukken/Onderdelen/pdf/binnen_de_wind.pdf}
		\caption{}
		\label{pic:binnen_wind}
	\end{minipage}
\end{figure}


\subsection{Overig}
Hiernaast dien je ook bekend te zijn met de onderstaande termen:

\begin{itemize}
	
	\item \textit{Bezeild}: Wanneer een punt bezeild kan worden, kan je hier komen zonder op te kruisen. 
	\item \textit{Planeren}: Wanneer een boot planeert vaart deze op zijn eigen boeggolf. De boot gaat niet meer door het water heen, maar glijdt er overheen. 
	\item \textit{Volvallen}: Volvallen definieert de boeg waarover je weg zeilt als je in de wind stil gelegen hebt. Als je volvalt over bakboord is je grootzeil aan bakboord bij het wegvaren.
	\item \textit{Verhalen}: Wanneer een schip verhaalt wordt, wordt deze over een korte afstand verplaatst door bomen, wrikken of door het gebruik van lijnen.
	\item \textit{Verlijeren of Drift}: Verlijeren of drift is de zijwaartse snelheid die een schip krijgt door de wind. 	
	\item \textit{Opschieter}: Bij een hogerwal aanleggen door vlak voor de kant de neus van de boot in de wind te draaien en zo al je snelheid te verliezen.
	\item \textit{Dwarspeiling}: Met een dwarspeiling maak je een inschatting van de koers die je zult varen na een overstag.
	\item \textit{Bijliggen}: Tijdens het bijliggen wordt de fok bak gezet en opgeloeft met het roer. Hierdoor dobbert de boot tussen in de wind en halve wind in. Het roer en fok kunnen eventueel vast worden gezet.
	\item \textit{Killen van het zeil}: Je laat dan expres een deel van je zeil minder wind vangen. Dit doe je door je zeil te vieren totdat alleen het achterlijk nog wind vangt.
	\item \textit{Bak}: Wanneer je je fok bak doet, zet je deze aan de hoge kant in plaats van de lage kant.
	\item \textit{Deinzen}: Dit is wanneer je achteruit dobbert met de neus van je boot in de wind.
\end{itemize}


\section{Bootonderdelen}
In figuur \ref{pic:vlet_nummers} is een tekening van een lelievlet te zien met maar liefst 88 gelabelde onderdelen. De namen van de onderdelen staan in tabel \ref{table:vletwel}. Alle onderdelen, behalve die met grijze nummers, moet je kennen.

\begin{table}[h!]
\centering
\caption{Vletonderdelen}

\setlength\extrarowheight{5pt} %Add height to center text vertically
\renewcommand{\arraystretch}{0.75} %Shrink total heigt to keep row same heigt
\newcommand{\tabhead}[1]{\cellcolor{ocre}{\color[HTML]{FFFFFF}\sffamily \textbf{#1}}}
\newcommand{\NIL}[1]{\cellcolor{not}{#1}}
\label{table:vletwel}

\input{Hoofdstukken/Onderdelen/bootonderdelen.tex}

\setlength\extrarowheight{0pt} %Reset
\renewcommand{\arraystretch}{1} %Reset

\end{table}
\newpage
\begin{figure}[h!]
    \centering
    \includegraphics[width=1.15\textwidth]{Hoofdstukken/Onderdelen/png/lelievlet_onderdelen.png}
    \caption{Tekening lelievlet met nummers \protect\footnotemark}
    \centering
    \label{pic:vlet_nummers}
\end{figure}
\footnotetext{\textit{Lelievlet\_onderdelen.png}, https://www.willibrordusgroep.nl/Images/upload/cwo/lelievlet\_onderdelen.png, Feb 2021.
}

\subsection{Extra onderdelen}
\label{ss:extra}
Niet alle onderdelen zijn uit te beelden in figuur \ref{pic:vlet_nummers}. Deze extra onderdelen zijn daarom hieronder toegelicht.
\vspace*{-0.2cm}
\begin{itemize}
	\item \textit{Voorsteven}: Voorkant van een boot of boeg.
	\item \textit{Sluiting}: Middel om lijnen mee te bevestigen. Voorbeeld: harpje of musketon haak.
	\item \textit{Kous}: Verstevigd metalen oog in het zeil.
	\item \textit{Kiel}: Steekt onder de bodem van het schip uit en voorkomt verlijeren en geeft stabiliteit.
	\item \textit{Halstalie}: Blok en lijn waarmee de halshoek naar beneden gespannen kan worden.
\end{itemize}

\newpage

\section{Conclusie}
Naast dat je nu bekend bent met de bootonderdelen uit tabel \ref{table:vletwel} en de extra onderdelen uit paragraaf \ref{ss:extra},  zijn dit al de zeiltermen uit de vorige paragrafen die je kent en begrijpt.
\begin{itemize}[label=]
\begin{multicols}{4}
	\item Bakboord
	\item Stuurboord
	\item Loefzijde
	\item Lijzijde
	\item Hoge kant
	\item Lage kant
	\item Hogerwal
	\item Lagerwal
	\item In de wind
	\item Aan de wind
	\item Halve wind
    \item Ruime wind
    \item Voor de wind
    \item Oploeven
    \item Afvallen
    \item Bovenwinds
    \item Benedenwinds
    \item Overstag gaan
    \item Opkruisen
    \item Korte slag
    \item Lange slag
    \item Gijpen
    \item Binnen de wind
    \item Bezeild
    \item Planeren
    \item Volvallen
    \item Verhalen
    \item Verlijeren
    \item Drift
    \item Opschieter
    \item Dwarspeiling
    \item Bijliggen
    \item Killen van het zeil 
    \item Bak
    \item Deinzen
\end{multicols}
\end{itemize}

\header{3}
\chapter{Veiligheid, Weer \& Vaarproblematiek}
\section{Inleiding}
Wanneer je wil gaan zeilen is het belangrijk dat dit veilig gebeurt. Om voor deze veiligheid te zorgen zijn een aantal punten van groot belang. Hierbij kan je denken aan een reddingsvest, kennis van het weer, kennis van je boot maar ook dat van andere boten. Al deze punten worden in dit hoofdstuk behandeld.
\section{Reddingsvest}
Een reddingsvest is een belangrijk onderdeel van de veiligheid aan boord. Er zijn 5 situaties waar je een reddingsvest aan moet:
\begin{enumerate}
\begin{multicols}{2}
    \item Als je boots het zegt
    \item Als de staf het zegt
    \item Als de waterpolitie het zegt 
    \item Als je het zelf wilt 
    \item Wanneer je een regenjas, regenbroek of kaplaarzen aan hebt
\end{multicols}
\end{enumerate}
Daarnaast zijn er een aantal strenge eisen aan reddingsvesten. Een reddingsvest moet:
\begin{itemize}%Nog checen
    \item Je binnen 15 seconden op je rug draaien
    \item Je mond 7 cm boven het water houden
    \item De tekst \textit{"Front"} aan de voorkant bevatten
    \item In het Nederlands gegevens over het drijfvermogen en maximaal gewicht van de drager bevatten
    \item De naam en het adres van de fabrikant bevatten
    \item Voorzien zijn van handvatten waar iemand mee uit het water getild kan worden
    \item Oranje of rood zijn.
\end{itemize}

\section{Omslaan}
Wanneer je boot is omgeslagen, \textbf{blijf je bij je boot}. Het is namelijk altijd gevaarlijker om te gaan zwemmen dan om bij je boot te blijven. Hier zijn een aantal redenen voor: ten eerste koel je veel minder snel af als je boven op je boot zit, of eraan hangt. Ook raak je zo minder vermoeid dan wanneer je zwemt. Daarnaast ben je makkelijker te vinden voor mensen die hulp willen bieden.
\section{Gedragsregels}
De belangrijkste en meest voorkomende gedragsregels zijn de volgende:
\begin{itemize}
    \item Houd de schippersgroet in ere
    \item Kom niet op andermans schip zonder toestemming
    \item Houd je schip en omgeving schoon
    \item Het is gebruikelijk om zeilwedstrijden voorrang te geven / te vermijden
\end{itemize}
\subsection*{Schippersgroet}
Op het water is het een gewoonte om als schippers (roergangers) onderling naar elkaar te zwaaien. Dit staat bekend als ''de schippersgroet''. Niet alleen is het een vorm van beleefdheid, maar je weet hierdoor ook zeker dat de schipper van het andere schip jou gezien heeft. 

\section{Weersinvloeden}
Wanneer je gaat varen is het weer van groot belang. Dit kan namelijk bepalen of het wel veilig is om het water op te gaan. Er zijn hierbij drie factoren die van belang zijn; Het soort boot, het soort vaarwater en de kennis en ervaring van je bemanning. Daarnaast mag je met CWO Kielboot II maar varen tot windkracht 4.

Tijdens het varen is het verstandig om goed te letten op een weersomslag. Een weersomslag betekent dat het weer heel snel verandert. Het zou dus heel hard kunnen gaan waaien, regenen of zelfs stormen. Er zijn een aantal kenmerken die dit aan kunnen geven. 
\begin{itemize}
    \begin{multicols}{2}
    \item Snel opkomende bewolking of wind
    \item Bloemkoolwolken
    \item Stilte voor de storm
    \item Plotselinge wind draaiing
    \end{multicols}
\end{itemize}
Ook zijn er nog twee belangrijke termen die met wind draaiing te maken hebben. Dit zijn: ruimen en krimpen. Wanneer de wind ruimt, draait deze met de richting van de wijzers van de klok mee. Een krimpende wind is een wind draaiing die tegen de richting van de klok in gaat. Krimpende wind wordt vaak geassocieerd met het verslechteren van het weer. Wanneer de wind dus sterk krimpt is het verstandig om het weer goed in de gaten te houden.


\section{Vaarproblematiek andersoortige schepen}
\subsection{Dodehoek \hfill \textit{Figuur \ref{pic:dodehoek}}}
Net zoals in het verkeer bij vrachtauto's, kunnen grote schepen een dode hoek hebben. De dode hoek is het deel rondom het schip dat vanuit de stuurhut niet gezien kan worden. Sommige schepen hebben door hun vorm een dodehoek rondom het hele schip, niet alleen aan de voorkant. Als je bijvoorbeeld te dicht naast een schip vaart, kan de stuurman je mogelijk niet zien!

\subsection{Zuiging \hfill \textit{Figuur \ref{pic:zuiging}}}
Grote schepen hebben last van zuiging. De voorkant van het schip duwt het water weg en dit wordt aan de zij- en achterkant weer aangezogen. Kleine boten en zwemmers kunnen mee- of ondergezogen worden. Blijf dus uit deze gebieden weg.

  \begin{center}
  \begin{minipage}[b]{0.35\textwidth}
    \begin{figure}[H]
        \includegraphics[width=\textwidth]{Hoofdstukken/Veiligheid/pdf/dode_hoek.pdf}
        \caption{Dodehoek}
        \label{pic:dodehoek}
    \end{figure}
  \end{minipage}
    \hspace{2cm}
  \begin{minipage}[b]{0.35\textwidth}
  \begin{figure}[H]
        \includegraphics[width=\textwidth]{Hoofdstukken/Veiligheid/pdf/zuiging.pdf}
        \caption{Zuiging}
        \label{pic:zuiging}
    \end{figure}
  \end{minipage}
  \end{center}

\newpage
\subsection{Diepgang  \hfill \textit{Figuur \ref{pic:diepgang}}}
Veel wateren hebben een vaargeul, dit is een dieper deel van het vaarwater. Soms is dit aangegeven met boeien of tonnen. Grote boten die zwaar beladen zijn kunnen soms alleen in dit deel van het water varen. Ze zullen misschien niet kunnen wijken voor je en jij zal daar rekening mee moeten houden.

\subsection{Verlijeren  \hfill \textit{Figuur \ref{pic:verlijeren}}}
Net als bij een zeilboot, kunnen ook grote motorschepen verlijeren. Als de wind van de zijkant komt, zal deze het schip opzij duwen. Om dit te corrigeren zal hij een beetje schuin gaan varen. Dit zorgt ervoor dat hij meer ruimte inneemt en minder wendbaar is. Geef deze schepen de ruimte. 

  \begin{center}
  \begin{minipage}[b]{0.35\textwidth}
    \begin{figure}[H]
        \includegraphics[width=\textwidth]{Hoofdstukken/Veiligheid/pdf/diepgang.pdf}
        \caption{Diepgang}
        \label{pic:diepgang}
    \end{figure}
  \end{minipage}
    \hspace{2cm}
  \begin{minipage}[b]{0.35\textwidth}
  \begin{figure}[H]
        \includegraphics[width=\textwidth]{Hoofdstukken/Veiligheid/pdf/verlijeren.pdf}
        \caption{Verlijeren}
        \label{pic:verlijeren}
    \end{figure}
  \end{minipage}
  \end{center}


\section{Conclusie}
Je hebt in dit hoofdstuk geleerd wat belangrijk is om veilig te zeilen. Zo zijn er regels voor reddingsvesten, een gedragscode en is het slim om goed op het weer te letten - zowel voor als tijdens het varen. Als laatst is er nog gekeken naar vaarproblemen bij andere, voornamelijk grote, schepen. 

\header{4}
\chapter{Bruggen \& Sluizen}
\section{Inleiding}
Bij langere tochten over het water zal je al snel te maken krijgen met bruggen en sluizen. In het BPR (Binnenvaartpolitiereglement) staan de regels voor het gebruik hiervan gedefinieerd. In dit hoofdstuk worden deze regels uitgelegd om zo veilig een brug of sluis te kunnen passeren.

\section{Vaste bruggen}
Bruggen zijn te onderscheiden in twee soorten: vaste en beweegbare bruggen. Een vaste brug, zoals in figuur \ref{pic:brug:vast}, kan niet open. Bij een beweegbare brug is een of meerdere wegdelen van de brug beweegbaar om grotere schepen te laten passeren. 

De brug in figuur \ref{pic:brug:vast} heeft drie vaste brugopeningen. De linker opening heeft een rood bord met een witte streep. Dit betekent dat doorvaart verboden is. De middelste opening heeft een enkele gele ruit. Dit betekent dat doorvaart toegestaan is, maar dat tegenliggende vaart mogelijk is. De rechter opening heeft twee gele ruiten. Dit betekent dat doorvaart is toegestaan en tegenliggende vaart verboden is. Aan de achterkant van deze vaaropening zal dan ook een `doorvaart verboden' bord hangen. 

Als je de keuze hebt tussen een of twee gele ruiten, maak dan altijd gebruik van de optie met de twee ruiten. Deze is het veiligst omdat je geen tegenliggers kunt hebben.
\begin{figure}[ht!]
  \centering
    \includegraphics[width=0.7\textwidth]{Hoofdstukken/Bruggen/pdf/brug_vast.pdf}
    \caption{}
    \label{pic:brug:vast}
\end{figure}

\section{Beweegbare bruggen}
Naast vaste bruggen zijn er ook beweegbare bruggen. Deze bruggen hebben lichten in plaats van borden. Vaak heeft een beweegbare brug naast een beweegbare opening, ook een vaste opening. Deze openingen beschikken dan ook over borden of lichten.  

\newpage

% --- Beweegbaar verboden door te varen
\begin{figure}[H]
  \centering
  \begin{minipage}[b]{0.18\textwidth}
    \includegraphics[width=\textwidth]{Hoofdstukken/Bruggen/pdf/brug_doorvaart_verboden.pdf}
    \caption{}
    \label{pic:brug:verboden}
  \end{minipage}
  \hfill
  \begin{minipage}[t]{0.75\textwidth}
  	\vspace{-2.5cm}
    Figuur \ref{pic:brug:verboden} betekent vrijwel hetzelfde als het rode bord uit figuur \ref{pic:brug:vast}. Doorvaart is verboden. Wanneer het echter de enige doorvaart is en je onder de gesloten brug past, mag je er wel door. Er kunnen dan ook tegenliggers aankomen.
  \end{minipage}
\end{figure}
% --- Beweegbaar doorvaart toegestaan, tegenliggende vaart mogelijk
\vspace{-0.75cm}
\begin{figure}[H]
	\centering
	\begin{minipage}[b]{0.18\textwidth}
		\includegraphics[width=\textwidth]{Hoofdstukken/Bruggen/pdf/brug_doorvaart_toegestaan.pdf}
		\caption{}
		\label{pic:brug:toegestaan}
	\end{minipage}
	\hfill
	\begin{minipage}[t]{0.75\textwidth}
	\vspace{-2.5cm}
	Figuur \ref{pic:brug:toegestaan} heeft dezelfde betekenis als een enkele gele ruit. De doorvaart is toegestaan, maar tegenliggende vaart is mogelijk. Wanneer je de optie hebt, kies dan voor de doorvaart met twee gele lichten. 
\end{minipage}
\end{figure}
% --- Beweegbaar doorvaart toegestaan, tegenliggende vaart niet mogelijk
\vspace{-0.75cm}
\begin{figure}[H]
\centering
\begin{minipage}[b]{0.18\textwidth}
	\includegraphics[width=\textwidth]{Hoofdstukken/Bruggen/pdf/brug_doorvaart_geen_tegenligger.pdf}
	\caption{}
	\label{pic:brug:toegestaan_tegenligger}
\end{minipage}
\hfill
\begin{minipage}[t]{0.75\textwidth}
	\vspace{-2.5cm}
	Figuur \ref{pic:brug:toegestaan_tegenligger} staat gelijk aan de twee gele ruiten. De doorvaart is toegestaan en tegenliggende vaart is niet mogelijk. Aan de andere kant van deze brug hangt een enkel rood licht of `verboden in te varen' bord.
\end{minipage}
\end{figure}
% --- Beweegbaar doorvaart aanstonds toegestaan
\vspace{-0.75cm}
\begin{figure}[H]
	\centering
	\begin{minipage}[b]{0.18\textwidth}
		\includegraphics[width=\textwidth]{Hoofdstukken/Bruggen/pdf/brug_aanstonds_toegestaan.pdf}
		\caption{}
		\label{pic:brug:aanstonds}
	\end{minipage}
	\hfill
	\begin{minipage}[t]{0.75\textwidth}
		\vspace{-2.5cm}
		Wanneer je niet onder een brug past en deze beweegbaar is, kan hij voor je opengaan. Wanneer een brug bijna open gaat, gaan de lichten branden als in figuur \ref{pic:brug:aanstonds}. Doorvaart is nog verboden totdat alleen het groene licht brandt
	\end{minipage}
\end{figure}
% --- Beweegbaar doorvaart toegestaan
\vspace{-0.75cm}
\begin{figure}[H]
	\centering
	\begin{minipage}[b]{0.18\textwidth}
		\includegraphics[width=\textwidth,]{Hoofdstukken/Bruggen/pdf/brug_toegestaan.pdf}
		\caption{}
		\label{pic:brug:vrij}
	\end{minipage}
	\hfill
	\begin{minipage}[t]{0.75\textwidth}
		\vspace{-2.5cm}
		Wanneer doorvaart door een beweegbare brug is toegestaan brandt, er een enkel groen licht zoals in figuur \ref{pic:brug:vrij}. Het kan ook zijn dat wanneer de brug open is, je eerst een enkel rood licht krijgt. Dit betekent dat de tegenliggers eerst mogen. Hierna zul jij een groen licht krijgen.
	\end{minipage}
\end{figure}
% --- Beweegbaar doorvaart aanstonds verboden
\vspace{-0.75cm}
\begin{figure}[H]
	\centering
	\begin{minipage}[b]{0.18\textwidth}
		\includegraphics[width=\textwidth]{Hoofdstukken/Bruggen/pdf/brug_sluitend.pdf}
		\caption{}
		\label{pic:brug:sluitend}
	\end{minipage}
	\hfill
	\begin{minipage}[t]{0.75\textwidth}
		\vspace{-2.5cm}
		Wanneer een brug bijna gaat sluiten of aan het sluiten is, gaat er een groen knipperend en rood licht branden, zoals in figuur \ref{pic:brug:sluitend}. De doorvaart is nu verboden, tenzij je redelijkerwijs niet meer kan stoppen. Dit is dus vergelijkbaar met een oranje verkeerslicht. 
	\end{minipage}
\end{figure}
% --- Beweegbaar buiten gebruik
\vspace{-0.75cm}
\begin{figure}[H]
	\centering
	\begin{minipage}[b]{0.18\textwidth}
		\includegraphics[width=\textwidth]{Hoofdstukken/Bruggen/pdf/brug_buiten_dienst.pdf}
		\caption{}
		\label{pic:brug:buiten}
	\end{minipage}
	\hfill
	\begin{minipage}[t]{0.75\textwidth}
		\vspace{-2.5cm}
		Als er een dubbel rood licht brandt (figuur \ref{pic:brug:buiten}), betekent het dat de brug buiten bediening is. De brugwachter kan dan bijvoorbeeld geen dienst hebben. Doorvaart is dan verboden. Wanneer er echter in het midden één of twee gele ruiten/lichten hangen gelden dezelfde regels als bij een enkel rood licht met gele ruit/licht.
	\end{minipage}
\end{figure}
% --- Beweegbaar gebied
\vspace{-0.75cm}
\begin{figure}[H]
	\centering
	\begin{minipage}[b]{0.18\textwidth}
		\includegraphics[width=\textwidth,]{Hoofdstukken/Bruggen/pdf/brug_aanbevolen_gebied.pdf}
		\caption{}
		\label{pic:brug:gebied}
	\end{minipage}
	\hfill
	\begin{minipage}[t]{0.75\textwidth}
		\vspace{-2.5cm}
		De ruiten in figuur \ref{pic:brug:gebied} geven iets aan over het aanbevolen vaargebied. Het is aanbevolen om binnen de groene ruiten te blijven varen. Dit kan te maken hebben met bijvoorbeeld een ondiepte of ander obstakel.
	\end{minipage}
\end{figure}
% --- Beweegbaar gebied verbod
\vspace{-0.75cm}
\begin{figure}[H]
\centering
\begin{minipage}[b]{0.18\textwidth}
	\includegraphics[width=\textwidth]{Hoofdstukken/Bruggen/pdf/brug_verboden_gebied.pdf}
	\caption{}
	\label{pic:brug:gebied_verbod}
\end{minipage}
\hfill
\begin{minipage}[t]{0.75\textwidth}
	\vspace{-2.5cm}
	Soms is het echter ook verboden om in bepaalde gebieden te varen. Dit wordt dan duidelijk gemaakt met de twee rode ruiten in figuur \ref{pic:brug:gebied_verbod}. Je moet dan tussen de rode ruiten in blijven en mag hier niet buiten varen.
\end{minipage}
\end{figure}

% --- Hoogteschaal
\vspace{-0.75cm}
\begin{figure}[H]
	\centering
	\begin{minipage}[b]{0.18\textwidth}
		\includegraphics[width=\textwidth]{Hoofdstukken/Reglementen/pdf/hoogteschaal.pdf}
		\caption{}
		\label{pic:brug:schaal}
	\end{minipage}
	\hfill
	\begin{minipage}[t]{0.75\textwidth}
		\vspace{-3cm}
		In sommige gevallen is het bord in figuur \ref{pic:brug:schaal} bij een brug te zien. Deze `hoogteschaal' kan gebruikt worden om de hoogte van de brug tot het water af te lezen. Je leest de hoogte af op het punt waar het water het bord raakt. De hoogte is uitgedrukt in meters.
	\end{minipage}
\end{figure}

\section{Sluizen}
Een sluis wordt gebruikt om een boot te verplaatsen tussen twee wateren met een verschillende hoogte. Wanneer je een sluis in mag varen wordt, net als bij bruggen, bepaald door lichten.
Bij sluizen hebben de lichten vrijwel exact dezelfde betekenis als bij bruggen. Er zijn echter ook wat kleine verschillen. 

Vaak hangen er in een de sluis zelf ook lichten. Deze maken duidelijk wanneer je de sluis uit mag varen. Wanneer er een sluiswachter aanwezig is moet je goed naar zijn instructies luisteren. Hij geeft vaak aan waar je moet gaan liggen in de sluis. 

% --- Sluis verbod
\hfill
\begin{figure}[H]
	\centering
	\begin{minipage}[b]{0.18\textwidth}
		\includegraphics[width=\textwidth]{Hoofdstukken/Bruggen/pdf/sluis_verboden.pdf}
		\caption{}
		\label{pic:sluis:verbod}
	\end{minipage}
	\hfill
	\begin{minipage}[t]{0.75\textwidth}
		\vspace{-2cm}
		Figuur \ref{pic:sluis:verbod} betekent net als bij bruggen dat doorvaart verboden is. Ook als de deuren helemaal open zijn, moet je wachten tot de lichten groen worden. Als er boten in de sluis liggen, moeten deze er namelijk eerst uit.
	\end{minipage}
\end{figure}
% --- Sluis verbod aanstonds
\vspace{-0.35cm}
\begin{figure}[H]
	\centering
	\begin{minipage}[b]{0.18\textwidth}	
		\includegraphics[width=\textwidth]{Hoofdstukken/Bruggen/pdf/sluis_aanstonds.pdf}
		\caption{}
		\label{pic:sluis:aanstonds}
	\end{minipage}
	\hfill
	\begin{minipage}[t]{0.75\textwidth}
		\vspace{-2cm}
		Wanneer de sluis bijna open gaat zullen de lichten aan gaan zoals in figuur \ref{pic:sluis:aanstonds}. Bij sommige sluizen is dit ook te zien als ze bijna gaan sluiten. Je mag er dan alleen nog in varen als je echt niet meer kan stoppen.
	\end{minipage}
\end{figure}
% --- Sluis toegestaan
\vspace{-0.35cm}
\begin{figure}[H]
	\centering
	\begin{minipage}[b]{0.18\textwidth}	
		\includegraphics[width=\textwidth]{Hoofdstukken/Bruggen/pdf/sluis_toegestaan.pdf}
		\caption{}
		\label{pic:sluis:toegestaan}
	\end{minipage}
	\hfill
	\begin{minipage}[t]{0.75\textwidth}
		\vspace{-2cm}
		Wanneer je de sluis in mag varen, geeft de sluis een enkel groen licht. Dit is te zien in figuur \ref{pic:sluis:toegestaan}. Wanneer de lichten groen zijn zullen alle boten die eerst in de sluis zaten, deze verlaten hebben. 
	\end{minipage}
\end{figure}
% --- Sluis buiten bedrijf
\vspace{-0.35cm}
\begin{figure}[H]
	\centering
	\begin{minipage}[b]{0.18\textwidth}	
		\includegraphics[width=\textwidth]{Hoofdstukken/Bruggen/pdf/sluis_buiten_dienst_dicht.pdf}
		\caption{}
		\label{pic:sluis:buiten}
	\end{minipage}
	\hfill
	\begin{minipage}[t]{0.75\textwidth}
		\vspace{-2cm}
		Een sluis kan net als een brug buiten bedrijf zijn. Dit wordt aangegeven met dubbele rode lichten uit figuur \ref{pic:sluis:buiten}. De deuren zullen in dit geval dicht zijn.
	\end{minipage}
\end{figure}
% --- Sluis buiten toegestaan
\vspace{-0.35cm}
\begin{figure}[H]
	\centering
	\begin{minipage}[b]{0.18\textwidth}	
		\includegraphics[width=\textwidth]{Hoofdstukken/Bruggen/pdf/sluis_buiten_dienst_open.pdf}
		\caption{}
		\label{pic:sluis:buiten_toegestaan}
	\end{minipage}
	\hfill
	\begin{minipage}[t]{0.75\textwidth}
		\vspace{-2cm}
		Het kan ook voorkomen dat de sluis buiten bedrijf is, maar doorvaart is toegestaan. Beide deuren staan dan open en de sluis geeft een dubbel groen licht, zie figuur \ref{pic:sluis:buiten_toegestaan}
	\end{minipage}
\end{figure}
\paragraph{Spuien en inlaten}
Om aan wachtende schepen de status van de sluis door te geven wordt er gebruikgemaakt van een aantal lichten en tekens. Deze tekens zijn optioneel en worden niet door alle sluizen gebruikt. Voor het lozen van water of `spuien' worden de tekens in figuur \ref{pic:sluis:spuien} gebruikt. Het inlaten van water wordt aangegeven met de tekens in figuur \ref{pic:sluis:inlaten}

  \begin{center}
	\begin{minipage}[b]{0.40\textwidth}
		\begin{figure}[H]
			\centering
			\includegraphics[width=0.46\textwidth]{Hoofdstukken/Bruggen/pdf/sluis_spuien.pdf}
			\caption{Spuien}
			\label{pic:sluis:spuien}
		\end{figure}
	\end{minipage}
	\hspace{1cm}
	\begin{minipage}[b]{0.40\textwidth}
		\begin{figure}[H]
			\centering
			\includegraphics[width=0.46\textwidth]{Hoofdstukken/Bruggen/pdf/sluis_inlaten.pdf}
			\caption{Inlaten}
			\label{pic:sluis:inlaten}
		\end{figure}
	\end{minipage}
\end{center}

\paragraph{Brug en sluis combinatie}
Het komt weleens voor dat er een sluis en brug direct naast elkaar geplaatst zijn. Let hierbij goed op de lichten. Het kan voorkomen dat je vrij lang voor de open brug moet wachten omdat de sluis eerst leeg moet varen. Wacht dan dus voor de brug, ook al pas je onder de brug door!

\section{Conclusie}
In dit hoofdstuk zijn alle lichten, tekens en regels voor bruggen en sluizen behandeld. Je weet nu wanneer het verboden en toegestaan is om een brug of sluis door te varen. Deze kennis is bijvoorbeeld heel erg van belang op een hike. Veel van de lichten hebben een logische betekenis en lijken soms zelfs een beetje op verkeerslichten. 

\header{5}
\chapter{Reglementen \& Voorrangsregels}
\section{Inleiding}
In dit hoofdstuk gaan we kijken naar de regels en wetten op het water. Op de meeste wateren waar jullie zullen varen wordt gebruikgemaakt van het Binnenvaartpolitiereglement, het BPR. Het BPR bevat alle regels over hoe je met elkaar om moet gaan op het water.

\section{Algemene reglementen}
Om het BPR goed te kunnen begrijpen, zullen we eerst een aantal algemene zaken bespreken. We beginnen met vier definities; motorschip, zeilschip, klein schip en groot schip. 

\begin{itemize}
    \item \textbf{Motorschip:} Een schip dat mechanische middelen gebruikt om zich voort te bewegen, een motor dus.
    \item \textbf{Zeilschip:} Een schip dat \textbf{alleen} zijn zeilen gebruikt om voort te bewegen. Hier onder valt ook een surfplank, maar niet een zeilboot met een motor aan.
    \item \textbf{Klein schip:} Alle schepen onder de 20 meter, met uitzondering van: passagiersschip \footnote{Draagt overdag een gele ruit achter op het schip om dit aan te duiden}, veerpont, visser, sleepboot
(alleen als deze grote schepen sleept), duwboot en duwbak. Deze uitzonderingen zijn altijd grote schepen. 
    \item \textbf{Groot schip:} Schepen groter dan 20 meter, inclusief de eerdergenoemde uitzonderingen 
\end{itemize}

\subsection{Goed zeemanschap}
Het goed zeemanschap is een hele belangrijke regel op het water. Deze regel houdt in dat de schipper bij het ontbreken van duidelijke regels \textbf{alle nodige voorzorgsmaatregelen} moet nemen om de veiligheid te garanderen, schade te voorkomen of de doorstroom op het water te versoepelen. Ook mag een schipper voor eigen veiligheid of die van anderen afwijken van het BPR.

\subsection{Andere reglementen}
Het BPR geldt niet op alle wateren. Het is verstandig om vooraf (als je op voor jou onbekende wateren gaat varen) uit te zoeken welke regels er gelden. Dit kan bijvoorbeeld in de ANWB Wateralmanak. In deel 1 staan alle regels en wetten die gelden in Nederland en België. 

\newpage
\section{Voorrangsregels}
Voorrangssituaties zijn te verdelen in drie types: kruisende koersen, tegengestelde koeren en oplopende koersen. Deze drie zijn te zien in figuur \ref{pic:voorrangkoers}. Welke van deze situaties je vaart bepaalt met welke regels je te maken hebt. 
\begin{figure}[H]
    \centering
    \includegraphics[width=0.8\textwidth]{Hoofdstukken/Reglementen/pdf/voorrangskoersen.pdf}
    \caption{Voorrangskoersen}
    \centering
    \label{pic:voorrangkoers}
\end{figure}
De voorrangsregels hebben ook een volgorde. Na het bepalen van welke koers je vaart, kijk je altijd eerst naar de bovenste regel die hierbij hoort. Als deze regel niet van toepassing is, ga je pas door naar de volgende. Dit doe je net zo lang totdat er een regel is die toe te passen is op jouw situatie.


\paragraph{Kruisende koersen}
\vspace{-0.7cm}
\begin{figure}[H]
	\centering
	\begin{minipage}[t]{0.70\textwidth}
		\textbf{1.} Het schip dat aan de stuurboordswal vaart heeft voorrang.\\
		\textit{Zeilschip A vaart aan de stuurboordswal en heeft voorrang op B}
	\end{minipage}
	\hfill
	\begin{minipage}[t]{0.20\textwidth}
		\raisebox{-0.5\height}{\includegraphics[width=\textwidth]{Hoofdstukken/Reglementen/pdf/kruis_stuurboordswal.pdf}}
		\label{pic:kr1}
	\end{minipage}
	\hfill
\end{figure}

\vspace{-0.7cm}

\begin{figure}[H]
	\centering
	\begin{minipage}[t]{0.70\textwidth}
		\textbf{2.} Grote schepen hebben voorrang op kleine schepen.\\
		\textit{Het grote motorschip A heeft voorrang op het kleine motorschip B}
	\end{minipage}
	\hfill
	\begin{minipage}[t]{0.20\textwidth}
		\raisebox{-0.5\height}{\includegraphics[width=\textwidth]{Hoofdstukken/Reglementen/pdf/kruis_groot_klein.pdf}}	
		\label{pic:kr2}
	\end{minipage}
	\hfill
\end{figure}

\vspace{-0.7cm}
\begin{figure}[H]
	\centering
	\begin{minipage}[t]{0.72\textwidth}
		\textbf{3.} Schepen op het hoofdvaarwater gaan voor op het nevenvaarwater.\\
		\textit{Motorschip A op het hoofdvaarwater heeft voorrang op motorschip B}
	\end{minipage}
	\hfill
	\begin{minipage}[t]{0.20\textwidth}
		\raisebox{-0.5\height}{\includegraphics[width=\textwidth]{Hoofdstukken/Reglementen/pdf/kruis_hoofd_neven.pdf}}	
		\label{pic:kr3}
	\end{minipage}
	\hfill
\end{figure}

\vspace{-0.7cm}
\begin{figure}[H]
	\centering
	\begin{minipage}[t]{0.70\textwidth}
		\textbf{4.} Een zeilschip gaat voor een roeiboot, gaat voor een motorschip.\\
		\textit{Zeilschip A heeft voorrang op roeiboot B en motorschip C\\
			Roeiboot B heeft voorrang op motorschip C}
	\end{minipage}
	\hfill
	\begin{minipage}[t]{0.20\textwidth}
		\raisebox{-0.65\height}{\includegraphics[width=\textwidth]{Hoofdstukken/Reglementen/pdf/kruis_zsm.pdf}}	
		\label{pic:kr4}
	\end{minipage}
	\hfill
\end{figure}

\vspace{-0.7cm}
\begin{figure}[H]
	\centering
	\begin{minipage}[t]{0.70\textwidth}
		\textbf{5.} Motor- en roeiboten onderling: Het schip van rechts gaat voor.\\
		\textit{Motorschip A op rechts heeft voorrang op motorschip B}
	\end{minipage}
	\hfill
	\begin{minipage}[t]{0.20\textwidth}
		\raisebox{-0.65\height}{\includegraphics[width=\textwidth]{Hoofdstukken/Reglementen/pdf/kruis_motor_onderling.pdf}}	
		\label{pic:kr5}
	\end{minipage}
	\hfill
\end{figure}

\vspace{-0.7cm}

\textbf{6.} Bij zeilschepen onderling zijn de volgende twee regels van belang:
\vspace{-0.5cm}
\begin{figure}[H]
	\centering
	\hspace{0.02\textwidth}
	\begin{minipage}[t]{0.70\textwidth}
		\textbf{6.1}. Een zeilschip met zeilen over bakboord heeft voorrang.\\
		\textit{Zeilschip B (met zijn zeilen over bakboord) heeft voorrang op \\zeilschip A (met zijn zeilen over stuurboord)}
	\end{minipage}
	\hfill
	\begin{minipage}[t]{0.20\textwidth}
		\raisebox{-0.6\height}{\includegraphics[width=\textwidth]{Hoofdstukken/Reglementen/pdf/kruis_zeilboot_onderling_bakboord.pdf}}	
		\label{pic:kr41}
	\end{minipage}
	\hfill
\end{figure}

\vspace{-0.7cm}
\begin{figure}[H]
	\centering
	\hspace{0.02\textwidth}
	\begin{minipage}[t]{0.70\textwidth}
		\textbf{6.2.} Een zeilschip aan loef wijkt voor een zeilschip aan lij.\\
		\textit{Zeilschip A ligt aan loef van zeilschip B en verleent dus voorrang}
	\end{minipage}
	\hfill
	\begin{minipage}[t]{0.20\textwidth}
		\raisebox{-0.55\height}{\includegraphics[width=\textwidth]{Hoofdstukken/Reglementen/pdf/kruis_zeilboot_onderling_loef_lij.pdf}}	
		\label{pic:kr42}
	\end{minipage}
	\hfill
\end{figure}

\paragraph{Tegengestelde koersen}
\vspace{-0.2cm}
\begin{figure}[H]
	\centering
	\begin{minipage}[t]{0.70\textwidth}
		\textbf{1.} Het schip dat aan de stuurboordswal vaart heeft voorrang.\\
		\textit{Zeilschip B vaart aan de stuurboordswal en heeft voorrang op A}
	\end{minipage}
	\hfill
	\begin{minipage}[t]{0.25\textwidth}
		\raisebox{-0.55\height}{\includegraphics[width=\textwidth]{Hoofdstukken/Reglementen/pdf/tegen_stuurboord.pdf}}
		\label{pic:tg1}
	\end{minipage}
	\hfill
\end{figure}
\vspace{-0.7cm}

\begin{figure}[H]
	\centering
	\begin{minipage}[t]{0.70\textwidth}
		\textbf{2.} Grote schepen hebben voorrang op kleine schepen.\\
		\textit{Het grote motorschip B heeft voorrang op het kleine motorschip A}
	\end{minipage}
	\hfill
	\begin{minipage}[t]{0.25\textwidth}
		\raisebox{-0.55\height}{\includegraphics[width=\textwidth]{Hoofdstukken/Reglementen/pdf/tegen_groot_klein.pdf}}
		\label{pic:tg2}
	\end{minipage}
	\hfill
\end{figure}
\vspace{-0.7cm}

\begin{figure}[H]
	\centering
	\begin{minipage}[t]{0.70\textwidth}
		\textbf{3.} Een zeilschip gaat voor een roeiboot, gaat voor een motorschip.\\
		\textit{Zeilschip A heeft voorrang op roeiboot B \\
			Zeilschip C heeft voorrang op motorschip D \\
			Roeiboot E heeft voorrang op motorschip F}
	\end{minipage}
	\hfill
	\begin{minipage}[t]{0.25\textwidth}
		\raisebox{-0.75\height}{\includegraphics[width=\textwidth]{Hoofdstukken/Reglementen/pdf/tegen_zsm.pdf}}
		\label{pic:tg3a}
	\end{minipage}
	\hfill
\end{figure}
\vspace{-0.7cm}

\begin{figure}[H]
	\centering
	\begin{minipage}[t]{0.70\textwidth}
		\textbf{4.} Zeilschepen onderling: Een zeilschip met zeilen over bakboord heeft voorrang.\\
		\textit{Zeilschip B (met zijn zeilen over bakboord) heeft voorrang op A}
	\end{minipage}
	\hfill
	\begin{minipage}[t]{0.25\textwidth}
		\raisebox{-0.75\height}{\includegraphics[width=\textwidth]{Hoofdstukken/Reglementen/pdf/tegen_zeilboot_onderling.pdf}}	
		\label{pic:tg4}
	\end{minipage}
	\hfill
\end{figure}
\vspace{-0.7cm}

\begin{figure}[H]
	\centering
	\begin{minipage}[t]{0.70\textwidth}
		\textbf{5.} Roei- of motorschepen onderling: Beide wijken naar stuurboord.\\
		\textit{Beide motorschepen wijken naar stuurboord}
	\end{minipage}
	\hfill
	\begin{minipage}[t]{0.25\textwidth}
		\raisebox{-0.55\height}{\includegraphics[width=\textwidth]{Hoofdstukken/Reglementen/pdf/tegen_motor_spier_onderling.pdf}}	
		\label{pic:tg5}
	\end{minipage}
	\hfill
\end{figure}

\paragraph{Oplopen}
Oplopen is \textit{enkel} toegestaan wanneer dit gedaan kan worden zonder gevaar voor andere schepen. Oplopen wordt voornamelijk langs bakboord gedaan. Het is echter ook toegestaan, wanneer de situatie hierom vraagt, om langs stuurboord op te lopen.

\begin{figure}[H]
	\centering
	\begin{minipage}[t]{0.70\textwidth}
		Zeilschepen onderling lopen elkaar op via de loefzijde. Hierdoor neem je de wind uit de zeilen van het opgelopen schip en gaat het oplopen sneller. Tijdens het oplopen mag medewerking verlangd worden van het opgelopen schip.
	\end{minipage}
	\hfill
	\begin{minipage}[t]{0.25\textwidth}
		\raisebox{-0.8\height}{\includegraphics[width=\textwidth]{Hoofdstukken/Reglementen/pdf/oplopen.pdf}}
		\label{pic:op}
	\end{minipage}
	\hfill
\end{figure}

\newpage
\subsection{Voorrangsregels op een rij}
Om de voorrangsregels makkelijk te kunnen onthouden staan ze hieronder samengevat:\\[0.1cm]
Bij \textbf{kruisende koersen} kijk je naar de volgende regels:
\vspace*{-0.15cm}
\begin{enumerate}
	\item Het stuurboordswal varende schip gaat voor
	\item Grote schepen gaan voor op kleine schepen
	\item Hoofdwater gaat voor nevenwater
	\item Zeilschip gaat voor roeiboot gaat voor motorschip
	\item Roei- of motorschepen onderling: het schip van rechts gaat voor
	\item Zeilschepen onderling: 
	\stepcounter{enumi}
	\begin{enumerate}
		\item [1.]Zeilen over bakboord gaat voor
		\item [2.]Loef wijkt voor lij
	\end{enumerate}
\end{enumerate}

Bij \textbf{tegengestelde koersen} kijk je naar de volgende regels:
\vspace*{-0.15cm}
\begin{enumerate}
	\item Het stuurboordswal varende schip gaat voor
	\item Grote schepen gaan voor op kleine schepen
	\item Zeilschip gaat voor roeiboot gaat voor motorschip
	\item Zeilschepen onderling: zeilen over bakboord gaat voor
	\item Roei- of motorschepen onderling: beiden wijken naar stuurboord
\end{enumerate}

Bij \textbf{oplopende koersen} wijkt de oploper uit. Het opgelopen schip kan indien nodig uitwijken.

\subsection{Toevoegingen}
\begin{figure}[H]
	\centering
	\begin{minipage}[t]{0.65\textwidth}
		Wanneer je vaarwater oversteekt, heb je geen voorrang. Andere schepen moeten hun koers en snelheid niet of nauwelijks hoeven aan te passen voor jouw manoeuvre. \\
		
		
		De regel `Stuurboordswal gaat voor' gaat over de stuurboordszijde van het \textbf{vaarwater}, niet over de echte wal. Boot A en B in figuur \ref{pic:SBwal} varen dus beide aan stuurboordswal, ondanks dat zij niet aan een fysieke wal varen.
	\end{minipage}
	\hfill
	\begin{minipage}[t]{0.30\textwidth}
		\raisebox{-0.9\height}{\includegraphics[width=\textwidth]{Hoofdstukken/Reglementen/pdf/sb_wal.pdf}}
		\caption{}
		\label{pic:SBwal}
	\end{minipage}
\end{figure}


\section{Conclusie}
Na het lezen van dit hoofdstuk heb je verstand van de voorrangsregels op het water. Een van de belangrijkste is het goed zeemanschap, wat inhoudt dat je alles doet om een gevaarlijke situatie of aanvaring te voorkomen. Daarnaast ken je de verschillende voorrangssituaties en volgorde en weet je hoe je de regels moet toepassen. 

\header{6}
\chapter{Schiemannen}
\section{Inleiding}
In dit hoofdstuk worden de knopen geleerd die belangrijk zijn tijdens het varen. Je moet de knopen kunnen maken en begrijpen wanneer en waarom je ze gebruikt. Onderaan dit hoofdstuk staat een tabel waar je knopen worden afgetekend door een instructeur. 
\section{De knopen}
\subsection{Halve steek \hfill \hspace{2 cm} \textit{Figuur \ref{pic:halve_steek}} } 
Een halve steek leg je wanneer je een touw vast wil leggen waar weinig kracht op komt. De halve steek is de basis voor veel knopen en steken.
\subsection{Slipsteek \hfill \textit{Figuur \ref{pic:slip_steek}}}
De slipsteek kan alleen gebruikt worden in situaties waar weinig kracht op de lijn komt. Het voordeel van een slipsteek is dat hij snel los te maken is.
\subsection{Achtknoop \hfill \textit{Figuur \ref{pic:achtknoop}}}
Een achtknoop wordt gebruikt om een verdikking in een touw te maken. Hiermee voorkom je bijvoorbeeld dat een touw door een blok schiet. 
\begin{figure}[h]
  \centering
  \begin{minipage}[b]{0.32\textwidth}
  \centering
    \includegraphics[width=0.8\textwidth]{Hoofdstukken/Schiemannen/pdf/halve_steek.pdf}
    \caption{Halve Steek}
    \label{pic:halve_steek}
  \end{minipage}
  \hfill
  \begin{minipage}[b]{0.32\textwidth}
    \centering
    \includegraphics[width=0.8\textwidth]{Hoofdstukken/Schiemannen/pdf/slip_steek.pdf}
    \caption{Slipsteek}
    \label{pic:slip_steek}
    \end{minipage}
  \hfill
  \begin{minipage}[b]{0.32\textwidth}
    \centering
    \includegraphics[width=0.8\textwidth]{Hoofdstukken/Schiemannen/pdf/achtknoop.pdf}
    \caption{Achtknoop}
    \label{pic:achtknoop}
  \end{minipage}
\end{figure}
\subsection{Platte knoop \hfill \textit{Figuur \ref{pic:platte_knoop}}}
Deze knoop is geschikt voor het verbinden van twee uiteinde van een touw van gelijke dikte. Deze knoop is niet geschikt voor situaties waar veel kracht op de lijn komt te staan. Hiervoor is een schootsteek beter geschikt.
\subsection{Schootsteek \hfill \textit{Figuur \ref{pic:schoot_steek}}}
Een schootsteek is geschikt om twee touwen van ongelijke dikte aan elkaar te maken. De knoop is ook geschikt voor touwen van gelijke dikte en kan veel kracht aan. Leg met het dikkere touw altijd de lus. Dat maakt de knoop makkelijker. 
\subsection{Mastworp \hfill \textit{Figuur \ref{pic:mastworp}}}
Deze knoop wordt veel gebruikt in pionieren en om je boot aan te leggen. De knoop trekt zichzelf strakker naarmate er meer kracht op komt. Daarnaast kun je een slipsteek op een mastworp leggen. Dit voorkomt dat de mastworp los kan schieten als er veel aan getrokken wordt. 
\begin{figure}[h]
  \centering
  \begin{minipage}[b]{0.32\textwidth}
  \centering
    \includegraphics[width=0.8\textwidth]{Hoofdstukken/Schiemannen/pdf/platteknoop.pdf}
    \caption{Platte Knoop}
    \label{pic:platte_knoop}
  \end{minipage}
  \hfill
  \begin{minipage}[b]{0.32\textwidth}
    \centering
    \includegraphics[width=0.8\textwidth]{Hoofdstukken/Schiemannen/pdf/schootsteek.pdf}
    \caption{Schootsteek}
    \label{pic:schoot_steek}
    \end{minipage}
  \hfill
  \begin{minipage}[b]{0.32\textwidth}
    \centering
    \includegraphics[width=0.8\textwidth]{Hoofdstukken/Schiemannen/pdf/mastworp.pdf}
    \caption{Mastworp}
    \label{pic:mastworp}
  \end{minipage}
\end{figure}
\subsection{Paalsteek \hfill \textit{Figuur \ref{pic:paal_steek}}} 
De paalsteek is bedoeld om een niet slippende lus in een touw te leggen. De lus is erg sterk, maar kan wel gemakkelijk weer losgehaald worden.
\subsection{Dubbele halve steek \hfill \textit{Figuur \ref{pic:dub_halve_steek}}}
Een dubbele halve steek is geschikt om lijnen strak aan een oog vast te maken. Dit is bijvoorbeeld handig als je aan wilt leggen met een meerpen. Je wikkelt eerst de lijn tweemaal om een oog en legt er vervolgens twee halve steken in. Dit maakt een mastworp. Je kan de eerste halve steek ook vervangen door een slipsteek.
\subsection{Een tros opschieten \hfill \textit{Figuur \ref{pic:opschieten}}}
Een tros opschieten is een manier om een lijn op te bergen zonder dat deze in de knoop raakt. Tijdens het opschieten maak je een aantal gelijke lussen. Aan het einde wikkel je de rest van het touw om de lussen en leg je een knoop als in het figuur. Opschieten staat ook wel bekend als opbossen.
\begin{figure}[h]
  \centering
  \begin{minipage}[b]{0.32\textwidth}
  \centering
    \includegraphics[width=\textwidth]{Hoofdstukken/Schiemannen/pdf/paalsteek.pdf}
    \caption{Paalsteek}
    \label{pic:paal_steek}
  \end{minipage}
  \hfill
  \begin{minipage}[b]{0.32\textwidth}
    \centering
    \includegraphics[width=\textwidth]{Hoofdstukken/Schiemannen/pdf/dubble_halve_steek.pdf}
    \caption{Dubbele halve steek}
    \label{pic:dub_halve_steek}
    \end{minipage}
  \hfill
   \begin{minipage}[b]{0.32\textwidth}
    \centering
    \includegraphics[width=\textwidth]{Hoofdstukken/Schiemannen/pdf/opbossen.pdf}
    \caption{Opschieten}
    \label{pic:opschieten}
    \end{minipage}
\end{figure}
\newpage
\subsection{Een kikker beleggen \hfill \textit{Figuur \ref{pic:kikker1}, \ref{pic:kikker2} \& \ref{pic:kikker3}}}
Wanneer je een kikker belegt, leg je een lijn vast op een kikker. Dit is nodig voor bijvoorbeeld het hijsen van het zeil. Belangrijk bij het beleggen van een kikker in een boot is dat je de ''eindlus'' aan de bovenzijde van de kikker legt. Anders kan deze er af vallen en de kikker losraken. Daarnaast moet je het lusje zo draaien dat het uiteinde weer in de richting van het vorige achtje gaat.
\begin{figure}[h]
  \centering
  \begin{minipage}[b]{0.32\textwidth}
  \centering
    \includegraphics[width=\textwidth]{Hoofdstukken/Schiemannen/pdf/kikker1.pdf}
    \caption{Kikker 8'tjes}
    \label{pic:kikker1}
  \end{minipage}
  \hfill
  \begin{minipage}[b]{0.32\textwidth}
    \centering
    \includegraphics[width=\textwidth]{Hoofdstukken/Schiemannen/pdf/kikker2.pdf}
    \caption{Kikker eind lus}
    \label{pic:kikker2}
    \end{minipage}
  \hfill
   \begin{minipage}[b]{0.32\textwidth}
    \centering
    \includegraphics[width=\textwidth]{Hoofdstukken/Schiemannen/pdf/kikker3.pdf}
    \caption{Kikker afknopen}
    \label{pic:kikker3}
    \end{minipage}
\end{figure}
\section{Conclusie}
Na het lezen van dit hoofdstuk en het oefenen met de knopen, snap je het nut en toepassing van de verschillende knopen. Ook kan je alle knopen zonder voorbeeld leggen. Een instructeur heeft dit in de onderstaande tabel afgetekend.
\vspace{2cm}
\begin{table}[H]
\centering
\caption{Aftekenen knopen}
\label{my-label}
\begin{tabular}{|l|l|l|}
\hline
\textbf{Knoop of Handeling}  & \textbf{Paraaf} & \textbf{Paraaf} \\ \hline
\textit{Halve Steek}         &                 &                 \\ \hline
\textit{Slipsteek}          &                 &                 \\ \hline
\textit{Achtknoop}           &                 &                 \\ \hline
\textit{Platte Knoop}        &                 &                 \\ \hline
\textit{Schootsteek}        &                 &                 \\ \hline
\textit{Mastworp}            &                 &                 \\ \hline
\textit{Paalsteek}           &                 &                 \\ \hline
\textit{Dubbele halve steek} &                 &                 \\ \hline
\textit{Tros opschieten}     &                 &                 \\ \hline
\textit{Kikker beleggen}     &                 &                 \\ \hline
\end{tabular}
\end{table}

\header{8}
\chapter{Krachten op het schip}
\section{Inleiding}

Om optimaal gebruik van de wind te kunnen maken, is het van belang om de krachten en hun effecten op het schip te begrijpen. Door deze kennis juist toe te passen is het mogelijk om sneller en efficiënter te zeilen. De juiste kennis van krachten is ook handig voor een aantal zeilmanoeuvres. 

\section{Krachten en koppels}
\label{par:krachten_uitleg}
In dit hoofdstuk wordt gebruikgemaakt van twee eenvoudige natuurkundige principes: krachten en koppels. De theorie hierachter wordt daarom kort toegelicht.

\begin{figure}[H]
	\centering
	\begin{minipage}[t]{0.75\textwidth}
		\paragraph{Kracht}
		Wanneer een kracht op een voorwerp gezet wordt kunnen er twee dingen gebeuren: het voorwerp verplaatst of het voorwerp vervormt. In dit hoofdstuk wordt alleen gekeken naar het verplaatsende effect van een kracht. Een kracht is gedefinieerd door twee componenten: een richting en een grootte/sterkte. Een voorbeeld van een kracht op een boot is te zien in figuur  \ref{pic:kracht}. 
	\end{minipage}
	\hfill
	\begin{minipage}[t]{0.22\textwidth}
		\raisebox{-0.9\height}{\includegraphics[width=\textwidth,]{Hoofdstukken/Krachten/pdf/kracht_op_vlet.pdf}}
		\RemoveLine
		\caption{}
		\label{pic:kracht}
	\end{minipage}
\end{figure} 

\paragraph{Krachten optellen en ontbinden}
Met krachten kan ook `gerekend' worden. Zo kunnen twee krachten die op hetzelfde voorwerp werken worden opgeteld. Een enkele kracht kan daarentegen worden opgedeeld in meerdere krachten met hetzelfde effect.

In figuur \ref{pic:kracht_som} zijn twee krachten te zien die op hetzelfde punt werken: blauw en rood. Door een parallellogram te maken van de krachten, is het mogelijk om de zwarte kracht te krijgen. Deze kracht heeft hetzelfde effect als de rode en blauwe kracht bij elkaar.

Het tegenovergestelde is ook mogelijk. Wanneer je een enkele kracht hebt, kun je deze splitsen in twee krachten met hetzelfde effect. In figuur \ref{pic:kracht_splits} is een enkele zwarte krachtpijl te zien. Deze wordt opgesplitst in de richting van rood en blauw.

Door een parallellogram om de zwarte kracht heen te tekenen, is het mogelijk lengte van de rode en blauwe kracht te bepalen. De zijdes vanuit de punt geven de krachten weer die hetzelfde effect hebben als de zwarte kracht.


\begin{figure}[H]
	\centering
	\begin{minipage}[b]{0.32\textwidth}
		\centering
		\includegraphics[width=0.85\textwidth]{Hoofdstukken/Krachten/pdf/kracht_optellen.pdf}
		\caption{Optellen}
		\label{pic:kracht_som}
	\end{minipage}
	\hfill
	\begin{minipage}[b]{0.32\textwidth}
		\centering
		\includegraphics[width=0.85\textwidth]{Hoofdstukken/Krachten/pdf/kracht_richtingen.pdf}
		\caption{Enkele kracht}
		\label{pic:kracht_splits}
	\end{minipage}
	\hfill
	\begin{minipage}[b]{0.32\textwidth}
		\centering
		\includegraphics[width=0.85\textwidth]{Hoofdstukken/Krachten/pdf/kracht_splitsen.pdf}
		\caption{Gesplitste kracht}
		\label{pic:kracht_splits2}
	\end{minipage}
\end{figure}


\paragraph{Koppels en armen}
Een koppel is de natuurkundige beschrijving voor de kracht die een voorwerp doet draaien. In het voorbeeld in figuur \ref{pic:koppel} is een balk met een draaipunt te zien. Door een kracht op het einde van de balk te zetten zal er een draaiing ontstaan. Het effect van de kracht is een koppel om het draaipunt.

\begin{figure}[H]
	\centering
	\begin{minipage}[b]{0.49\textwidth}
		\centering
		\includegraphics[width=0.75\textwidth]{Hoofdstukken/Krachten/pdf/kracht_koppel.pdf}
		\caption{Koppel}
		\label{pic:koppel}
	\end{minipage}
	\hfill
	\begin{minipage}[b]{0.49\textwidth}
		\centering
		\includegraphics[width=0.75\textwidth]{Hoofdstukken/Krachten/pdf/kracht_arm.pdf}
		\caption{Arm}
		\label{pic:arm}
	\end{minipage}
\end{figure}
De afstand tussen de kracht en het draaipunt wordt ook wel de arm genoemd. Door de arm te vergroten is het mogelijk om met dezelfde kracht, een sterkere koppel te maken. Dit is weergegeven in figuur \ref{pic:arm}. De arm versterkt hier de kracht die op de balk geplaatst wordt.

\section{Voortstuwing van de boot}
Nu de basiskennis over de krachten is verworven, kan er verder gekeken worden naar hoe een boot voortgestuwd wordt. Dit wordt in drie stappen toegelicht: voortstuwende werking van de zeilen, driftbeperkende middelen en tot slot voortstuwing.

\subsection*{Voortstuwende werking van de zeilen}
De voortstuwende werking van de zeilen heeft alles te maken met de stroming van de wind langs de zeilen. In figuur \ref{pic:stroming} is een versimpeling van deze stroming te zien langs de fok en het grootzeil. 

Door de bolle vorm van de zeilen moet de wind aan de bolle zijde een langere weg afleggen dan aan de holle zijde. Hierdoor zal de wind aan de bolle zijde sneller stromen dan de holle zijde. Sneller stromende lucht heeft een lage druk. Hierdoor ontstaat een drukverschil zoals te zien is in figuur \ref{pic:druk}.

\begin{figure}[H]
	\centering
	\begin{minipage}[b]{0.32\textwidth}
		\centering
		\includegraphics[width=0.95\textwidth]{Hoofdstukken/Krachten/pdf/voortstuwing_zeil_stroming.pdf}
		\caption{Stroming}
		\label{pic:stroming}
	\end{minipage}
	\hfill
	\begin{minipage}[b]{0.32\textwidth}
		\centering
		\includegraphics[width=0.95\textwidth]{Hoofdstukken/Krachten/pdf/voortstuwing_zeil_druk.pdf}
		\caption{Druk}
		\label{pic:druk}
	\end{minipage}
	\hfill
	\begin{minipage}[b]{0.32\textwidth}
		\centering
		\includegraphics[width=0.95\textwidth]{Hoofdstukken/Krachten/pdf/voortstuwing_zeil_krachten_los.pdf}
		\caption{Voortstuwing}
		\label{pic:stuwing}
	\end{minipage}
\end{figure}

Omdat de lucht van het hoge naar het gebied met lage druk toe wil, ontstaat er een kracht op beide zeilen. Wanneer je per zeil alle krachten opsomt, krijg je de krachten te zien in figuur \ref{pic:stuwing}.

\begin{figure}[H]
	\centering
	\begin{minipage}[t]{0.63\textwidth}
		\vspace*{0.2cm}
		Om makkelijk met deze krachten te werken, nemen we de kracht van zowel de fok als het grootzeil samen. Dit levert de kracht in figuur \ref{pic:zeilpunt} op en heet de zeilkracht. Het punt waaruit de zeilkracht werkt noemen we het zeilpunt. 
	\end{minipage}
	\hfill
	\begin{minipage}[t]{0.32\textwidth}
		\raisebox{-0.9\height}{\includegraphics[width=0.95\textwidth]{Hoofdstukken/Krachten/pdf/voortstuwing_zeil_zeilkracht.pdf}}
		\RemoveLine
		\caption{Zeilpunt}
		\label{pic:zeilpunt}
	\end{minipage}
\end{figure} 

\subsection{Driftbeperkende middelen}
Om te voorkomen dat de zeilkracht de boot enkel verlijerd of drift, heeft deze driftbeperkende middelen. Deze middelen maken het lastiger voor de boot om zijwaarts over het water te bewegen doordat deze dwars op de boot staan. Een lelievlet beschikt over vier driftbeperkende middelen: het onderwaterschip, het zwaard, de scheg en het roer. In figuur \ref{pic:drift_beperking} zijn deze onderdelen afgebeeld.

\begin{figure}[H]
	\centering
	\begin{minipage}[b]{0.48\textwidth}
		\centering
		\includegraphics[width=0.90\textwidth]{Hoofdstukken/Krachten/pdf/lateraal_driftbeperking.pdf}
		\caption{Driftbeperking}
		\label{pic:drift_beperking}
	\end{minipage}
	\hfill
	\begin{minipage}[b]{0.48\textwidth}
		\centering
		\includegraphics[width=0.90\textwidth]{Hoofdstukken/Krachten/pdf/lateraal_boven.pdf}
		\caption{Zijwaartse kracht}
		\label{pic:zijwaarts}
	\end{minipage}
\end{figure}
Om de driftbeperking en de krachten die hierbij komen kijken beter te begrijpen, kijken we naar de volgende situatie: een boot wordt door een externe kracht zijwaarts over het water geduwd. Dit is uitgebeeld in figuur \ref{pic:zijwaarts}. De zwarte pijl stelt het duwen van de boot voor. 

Wanneer de boot zijwaarts verplaatst wordt, zullen de driftbeperkende middelen dit tegengaan omdat deze het water om zich heen moeten verplaatsen. In figuur \ref{pic:zijwaarts} is de weerstand die deze middelen geven met de rode pijlen uitgebeeld. Figuur \ref{pic:zijwaarts_totaal} toont een zijaanzicht van deze situatie.

\begin{figure}[H]
	\centering
	\begin{minipage}[b]{0.48\textwidth}
		\centering
		\includegraphics[width=0.90\textwidth]{Hoofdstukken/Krachten/pdf/lateraal_zij.pdf}
		\caption{Zijwaartse kracht zijaanzicht}
		\label{pic:zijwaarts_totaal}
	\end{minipage}
	\hfill
	\begin{minipage}[b]{0.48\textwidth}
		\centering
		\includegraphics[width=0.90\textwidth]{Hoofdstukken/Krachten/pdf/lateraal_punt.pdf}
		\caption{Lateraalpunt}
		\label{pic:lateraal}
	\end{minipage}
\end{figure}

Wanneer we al deze krachten sommeren krijgen we een totaal kracht (rood) die is afgebeeld in figuur \ref{pic:lateraal}. Deze kracht werkt vanuit een punt dat ook wel het lateraalpunt genoemd wordt. Het lateraalpunt is formeel gedefinieerd als: ``\textit{Het punt waar alle laterale (zijwaartse) krachten van het schip op werken}''. Het lateraalpunt is tevens het draaipunt van het schip.

Door middel van het lateraalpunt kan makkelijk gerekend worden met de driftbeperkende kracht van het schip. Dit zal van pas komen in het begrijpen van de voortstuwing van de boot.

\newpage
\subsection{Voortstuwing}
Met de kennis die is opgedaan over zowel de zeil- als driftbeperkende kracht is het mogelijk om te kijken naar hoe de boot vooruit bewogen  wordt. In figuur \ref{pic:krachten_los} is een boot te zien samen met een zeilkracht (blauw) en een driftbeperkende kracht (rood). Deze krachten zijn getekend vanuit het zeilpunt en het lateraalpunt.

\begin{figure}[H]
	\centering
	\begin{minipage}[b]{0.48\textwidth}
		\centering
		\includegraphics[width=0.90\textwidth]{Hoofdstukken/Krachten/pdf/voortstuwing_krachten.pdf}
		\caption{Zeil en driftbeperkende kracht}
		\label{pic:krachten_los}
	\end{minipage}
	\hfill
	\begin{minipage}[b]{0.48\textwidth}
		\centering
		\includegraphics[width=0.90\textwidth]{Hoofdstukken/Krachten/pdf/voortstuwing_totaal.pdf}
		\caption{Voortstuwende kracht}
		\label{pic:voorwaardse_kracht}
	\end{minipage}
\end{figure}

In figuur \ref{pic:voorwaardse_kracht} wordt de zeilkracht gesplitst in twee krachten met hetzelfde effect: de voorwaartse kracht (groen) en de verlijerende kracht (rood). De verlijerende kracht wordt echter opgeheven door de driftbeperkende kracht. Hierdoor blijft alleen de voortstuwende kracht nog over. Deze samenwerking van de krachten geeft een zeilboot zijn voortstuwing.

\section{Correcte zeilstand}
Door de zeilen in een correcte stand te zetten is het mogelijk de voortstuwende werking hiervan te optimaliseren. Dit heet ook wel het trimmen van de zeilen. Een correct getrimd zeil is weergegeven in figuur \ref{pic:zeil_goed}. In dit figuur is te zien dat de stroming van de wind de zeilen strak volgt. Deze stroming creëert zo het grootste drukverschil en hiermee de meeste voortstuwing. 

Wanneer het zeil echter te strak staat, ontstaat de situatie in figuur \ref{pic:zeil_strak}. De windstroming laat halverwege het grootzeil `los'. Dit zorgt voor turbulentie aan het achterlijk van het zeil waardoor deze gaat klapperen. De verstoring in de stroming en de turbulentie verlagen het drukverschil tussen beide zijden van het zeil. Dit zorgt vervolgens voor een lagere voortstuwing.

\begin{figure}[H]
  \centering
  \begin{minipage}[b]{0.32\textwidth}
  \centering
    \includegraphics[height=5cm]{Hoofdstukken/Krachten/pdf/trimmen_goed.pdf}
    \caption{Zeil goed}
    \label{pic:zeil_goed}
  \end{minipage}
  \hfill
  \begin{minipage}[b]{0.32\textwidth}
    \centering
    \includegraphics[height=5cm]{Hoofdstukken/Krachten/pdf/trimmen_strak.pdf}
    \caption{Zeil te strak}
    \label{pic:zeil_strak}
    \end{minipage}
  \hfill
  \begin{minipage}[b]{0.32\textwidth}
    \centering
    \includegraphics[height=5cm]{Hoofdstukken/Krachten/pdf/trimmen_los.pdf}
    \caption{Zeil te los}
    \label{pic:zeil_los}
  \end{minipage}
\end{figure}

In figuur \ref{pic:zeil_los} is een te los zeil te zien. De windstromen aan de hoge kant van het zeil volgen het zeil niet strak en aan de lage zijde wordt de stroming omgeleid door het zeil. Deze verstoringen in de stroming zorgen voor een lager drukverschil en minder voortstuwing. 

Een goede manier om je zeil te trimmen is om hem net zo lang te laten vieren totdat er een kleine tegenbolling in het voorlijk te zien valt. Daarna trek je het zeil weer een klein beetje aan tot deze weg valt. Op deze manier benut je de wind maximaal. Door dit regelmatig te doen weet je zeker dat je optimaal zeilt. 

\section{Effecten van de fok en het grootzeil}
Het grootzeil en de fok hebben allebei een ander effect op de koers van de boot. Het verschil in gedrag is te verklaren door hun positie ten opzichte van het lateraalpunt. Omdat het grootzeil (en het bijbehorende zeilpunt) zich achter het lateraalpunt bevindt, creëert het grootzeil een oploevend koppel. Dit effect is weergegeven in figuur \ref{pic:effect_grootzeil}.

Het omgekeerde is waar voor de fok. Doordat deze zich voor het lateraalpunt bevindt, ontstaat er een afvallend koppel. Figuur \ref{pic:effect_fok} toont deze situatie. 

\begin{figure}[ht]
	\centering
	\begin{minipage}[b]{0.49\textwidth}
		\centering
		\includegraphics[width=\textwidth]{Hoofdstukken/Krachten/pdf/effect_grootzeil.pdf}
		\caption{Effect van het grootzeil}
		\label{pic:effect_grootzeil}
	\end{minipage}
	\hfill
	\begin{minipage}[b]{0.49\textwidth}
		\centering
		\includegraphics[width=\textwidth]{Hoofdstukken/Krachten/pdf/effect_fok.pdf}
		\caption{Effect van de fok}
		\label{pic:effect_fok}
	\end{minipage}
\end{figure}

Door gebruik te maken van de sturende werking van de zeilen is het mogelijk de boot van koers te veranderen met minder gebruik van het roer. Hiermee wordt het oploeven en afvallen efficiënter.

\section{Effect van de helling}
Een lelievlet is van nature loefgierig. Dit houdt in dat de boot met een normale zeilstand, maar zonder roer, de wind in draait. Door de helling van de boot te veranderen is het mogelijk om de loef- en lijgierigheid hiervan aan te passen.

De loefgierigheid van de boot komt voort uit de positie van het zeilpunt ten opzichte van het lateraalpunt. Dit is duidelijk te zien in het achteraanzicht van een boot in figuur \ref{pic:helling_vlak_achter}. Tussen het zeilpunt en het lateraalpunt bevindt zich een arm. Omdat het lateraalpunt tevens het draaipunt van de boot is, zorgt deze arm voor een koppel om het lateraalpunt waardoor de boot wil oploeven. Dit koppel is weergegeven in figuur \ref{pic:helling_valk_boven}. De arm tussen het zeilpunt en het lateraalpunt geeft de boot dus zijn loefgierheid.

\begin{figure}[H]
	\centering
	\begin{minipage}[b]{0.48\textwidth}
		\centering
		\includegraphics[height=5cm]{Hoofdstukken/Krachten/pdf/helling_vlak_achter.pdf}
		\caption{Zeilpunt en lateraalpunt}
		\label{pic:helling_vlak_achter}
	\end{minipage}
	\hfill
	\begin{minipage}[b]{0.48\textwidth}
		\centering
		\includegraphics[height=5cm]{Hoofdstukken/Krachten/pdf/helling_vlak_boven.pdf}
		\caption{Oploevend koppel}
		\label{pic:helling_valk_boven}
	\end{minipage}
\end{figure}


De helling van de boot heeft een invloed op de lengte van deze arm. Door de boot meer naar de lijkant te hellen wordt de arm vergroot. Dit zorgt vervolgens voor een groter oploevend koppel. Figuur \ref{pic:helling_oploeven} laat deze vergrootte arm zien. 

\begin{figure}[H]
	\centering
	\begin{minipage}[b]{0.48\textwidth}
		\centering
		\includegraphics[height=5cm]{Hoofdstukken/Krachten/pdf/helling_lij_achter.pdf}
		\includegraphics[height=5cm]{Hoofdstukken/Krachten/pdf/helling_lij_boven.pdf}
		\caption{Helling naar lij}
		\label{pic:helling_oploeven}
	\end{minipage}
	\hfill
	\begin{minipage}[b]{0.48\textwidth}
		\centering
		\includegraphics[height=5cm]{Hoofdstukken/Krachten/pdf/helling_loef_achter.pdf}
		\includegraphics[height=5cm]{Hoofdstukken/Krachten/pdf/helling_loef_boven.pdf}
		\caption{Helling naar loef}
		\label{pic:helling_afvallen}
	\end{minipage}
\end{figure}

Het tegenovergestelde is ook mogelijk: door de boot te hellen naar de loefzijde, wordt de arm verkleind of zelfs omgedraaid. Dit zorgt ervoor dat de boot lijgierig wordt. Dit wordt weergegeven in figuur \ref{pic:helling_afvallen}.


Door correct gebruik te maken van de helling van het schip, is het mogelijk sneller en efficiënter af te vallen en op te loeven. Tijdens complexe manoeuvres biedt dit extra controle over de boot.  

\section{Stabiliteit}
De zeilkracht die de boot voortstuwt heeft ook een nadelige bijwerking: de zeilkracht wil de boot ook om duwen. De zeilkracht creëert een koppel rondom het lateraalpunt, genaamd het kenterend koppel. Het koppel is weergegeven in figuur \ref{pic:kenterende_koppel}. Een boot slaat echter niet zomaar om omdat deze stabiel is. Deze stabiliteit komt voort uit een evenwicht tussen kenterend koppel en zijn tegenhanger het oprichtend koppel. 

Het oprichtend koppel ontstaat vanuit twee krachten die werken vanuit twee punten:
\begin{enumerate}
	\item \textbf{Drukpunt:} Dit is het aangrijppunt van alle opwaartse kracht op de boot. Ook wel het drijfpunt genoemd.
	\item \textbf{Zwaartepunt:} Dit is het aangrijppunt voor alle zwaartekracht die op de boot werkt.
\end{enumerate}

In figuur \ref{pic:oprichtend} zijn de punten met hun krachten geïllustreerd. Het effect hiervan is het oprichtende koppel. Wanneer de boot verder helt, zoals in figuur \ref{pic:oprichtend_helling} is te zien dat de arm tussen het zwaartepunt en het drukpunt groter wordt. Hierdoor wordt ook het oprichtend koppel verstrekt. 

De boot in figuur \ref{pic:oprichtend_helling} is een gewichtsstabiele boot. Deze categorie boten hebben een laag zwaartepunt door hun bouw of een kiel. Deze boten hebben bij een kleine helling een lage stabiliteit. Naarmate de helling toeneemt, neemt de stabiliteit ook toe. 

\begin{figure}[H]
	\centering
	\begin{minipage}[b]{0.32\textwidth}
		\centering
		\includegraphics[height=6cm]{Hoofdstukken/Krachten/pdf/stabiliteit_kenternd.pdf}
		\caption{Kenterend koppel}
		\label{pic:kenterende_koppel}
	\end{minipage}
	\hfill
	\begin{minipage}[b]{0.33\textwidth}
		\centering
		\includegraphics[height=6cm]{Hoofdstukken/Krachten/pdf/stabiliteit_gewichtstabiel.pdf}
		\caption{Oprichtend koppel}
		\label{pic:oprichtend}
	\end{minipage}
	\hfill
	\begin{minipage}[b]{0.32\textwidth}
		\centering
		\includegraphics[height=6cm]{Hoofdstukken/Krachten/pdf/stabiliteit_gewichtstabiel_helling.pdf}
		\caption{Helling}
		\label{pic:oprichtend_helling}
	\end{minipage}
\end{figure}

De boot in figuur \ref{pic:vormstabiel} is een vormstabiele boot. Deze wordt gekenmerkt door een hoog zwaartepunt. Dit type schepen heeft een hoge stabiliteit bij een kleine helling. Naarmate de helling toeneemt, zoals in figuur \ref{pic:vormstabiel_helling}, komt het drukpunt dichter naar het zwaartepunt toe. Dit verkleint de arm en hierdoor neemt het oprichtend koppel af en dus ook de stabiliteit.
\begin{figure}[H]
	\centering
	\begin{minipage}[b]{0.49\textwidth}
		\centering
		\includegraphics[height=6cm]{Hoofdstukken/Krachten/pdf/stabiliteit_vormstabiel.pdf}
		\caption{Vormstabiel}
		\label{pic:vormstabiel}
	\end{minipage}
	\hfill
	\begin{minipage}[b]{0.49\textwidth}
		\centering
		\includegraphics[height=6cm]{Hoofdstukken/Krachten/pdf/stabiliteit_vormstabiel_helling.pdf}
		\caption{Vormstabiel onder helling}
		\label{pic:vormstabiel_helling}
	\end{minipage}
\end{figure}

Stabiliteit van schepen wordt dus gedicteerd door de ligging van het zwaartepunt ten opzichte van het drukpunt. Gewichtsstabiele schepen hebben daarom en lage beginstabiliteit en een hoge eindstabiliteit. Voor vormstabiele schepen is dit andersom. Deze hebben een hoge beginstabiliteit en een lage eindstabiliteit.  

\newpage
\section{Schijnbare wind}
De wind die op een boot werkt bestaat in feite uit twee delen: de ware wind en de vaartwind.

\begin{enumerate}
	\item De \textbf{ware wind} ontstaat door drukverschillen in de atmosfeer en voel je als je stil staat. De ware wind is ook af te zien aan vlaggen. 
	\item De \textbf{vaartwind} ontstaat daarentegen door de voorwaartse snelheid van het schip. Dit fenomeen is ook te voelen als je op een windstille dag fiets: door de voortbeweging voel je toch wind. 
\end{enumerate}

De ware wind en de vaartwind vormen samen de schijnbare wind. Dit is de wind waar het schip op vooruit gaat en de wind die je in de boot voelt. In figuur \ref{pic:schijnbare_wind} is de schijnbare wind te zien en hoe deze ontstaat vanuit zijn twee componenten. 

\begin{figure}[H]
	\centering
	\begin{minipage}[b]{0.49\textwidth}
		\centering
		\includegraphics[height=6cm]{Hoofdstukken/Krachten/pdf/wind_schijnbaar.pdf}
		\caption{Schijnbare wind}
		\label{pic:schijnbare_wind}
	\end{minipage}
	\hfill
	\begin{minipage}[b]{0.49\textwidth}
		\centering
		\includegraphics[height=6cm]{Hoofdstukken/Krachten/pdf/wind_vlaag.pdf}
		\caption{Windvlaag}
		\label{pic:windvlaag}
	\end{minipage}
\end{figure}

De samenwerking van de ware wind en de vaartwind geeft een bijzondere dynamiek aan de schijnbare wind. Wanneer de ware wind plots krachtiger wordt, bijvoorbeeld door een windvlaag, zal de wind wat `ruimer' de boot in komen. Dit is te zien in figuur \ref{pic:windvlaag}. Doordat de wind wat ruimer is, is het mogelijk wat hoger te varen. Zo is het mogelijk om tijdens een windvlaag wat hoogte te winnen.

Een tweede effect is dat wanneer de snelheid van het schip toeneemt, en dus ook de vaartwind, de wind juist scherper de boot in komt. Snel varende schepen kunnen daarom wat minder hoog aan de wind varen.

\section{Conclusie}
In dit hoofdstuk zijn de verschillende krachten op het schip behandeld. Je snapt nu onder andere hoe een schip aan zijn voortstuwing komt, wat het effect van het grootzeil en de fok is en wat de invloed van de helling op de boot is. Met deze kennis is het mogelijk meer controle over de boot te krijgen.  


\makeatletter\@addtoreset{chapter}{part}
\makeatother

\part{Oefenvragen}
\label{part:oefen}
\thispagestyle{empty}
\makeatletter\@openrightfalse
\makeatother
\setcounter{chapter}{0}
\usechapterimagefalse
\chapter{Bootonderdelen \& Zeiltermen}
\vspace{-120px}

\question{1}{Hoe heet een verstevigd metalen oog in het zeil?}
\answerTextFour{Splitsring}{Kous}{Lijoog}{Lus}

\question{2}{Een vaarwater is niet bezeild wanneer:}
\answerTextFour{Je moet opkruisen om je doel te bereiken}{Je niet hoeft op te kruisen om je doel te bereiken}{Je alleen op je fok naar je doel kunt varen}{Je aan stuurboordswal naar je doel kunt varen}

\question{3}{Wanneer je binnen de wind vaart:}
\answerTextFour{Vaar je ruime wind met je zeilen aan de verkeerde kant}{Ben je aan het opkruisen}{Bezeil je een punt dat in de wind ligt}{Vaar je exact voor de wind}

\question{4}{Hoe heet de hoek van de fok aangegeven in het plaatje?}
\vspace*{-0.75cm}
\answerTextPicture{Tophoek}{Halshoek}{Klauwhoek}{Schoothoek}{Hoofdstukken/Oefenvragen/pdf/schoothoek.pdf}


\question{5}{Welk onderdeel wijst de pijl aan?}
\vspace*{-0.65cm}
\answerTextPicture{Roerkoning}{Roerstaf}{Roerpen}{Vingerlingen}{Hoofdstukken/Oefenvragen/pdf/roerkoning.pdf}

\question{6}{Welke stelling is waar?}
\answerTextPicture{I is de hogerwal en III is de lage kant}{II is de hogerwal en III is de lagerwal}{IV is de lagerwal en II is de hogerwal}{II is de lage kant en III is de hoge kant}{Hoofdstukken/Oefenvragen/pdf/wallen.pdf}

\question{7}{Met welk lijntje is het zeil bevestigd aan de giek en gaffel}
\answerTextFour{Gaffeldraad}{Lijkentouw}{Rijglijn}{Marllijn}
\newpage
\question{8}{Op welke afbeelding vaart de boot ruime wind?}
\answerPicture{Hoofdstukken/Oefenvragen/pdf/aan_de_wind.pdf}{Hoofdstukken/Oefenvragen/pdf/voor_de_wind.pdf}{Hoofdstukken/Oefenvragen/pdf/ruime_wind.pdf}{Hoofdstukken/Oefenvragen/pdf/halve_wind.pdf}


\question{9}{Welke stelling is waar?}%boven en benede winds
\answerTextPicture{De boot rondt de boei bovenwinds over bakboord}{De boot rondt de boei bovenwinds over stuurboord}{De boot rondt de boei benedenwinds over bakboord}{De boot rondt de boei benedenwinds over stuurboord.}{Hoofdstukken/Oefenvragen/pdf/bovenwinds_over_stuurboord_ronden.pdf}

\question{10}{Wat gebeurt er met je boot als je gaat afvallen?}%boven en benede winds
\answerTextFour{Je boot draait van de wind af}{Je draait door de wind heen}{Je draait naar de wind toe}{Je boot ligt in de wind}



\chapter{Veiligheid, Weer \& Vaarproblematiek}
\vspace{-120px}

\question{1}{Wat is geen eis aan een reddingsvest?}
\answerTextFour{Oranje of rood van kleur zijn}{Naam en adres van de fabrikant bevatten}{Handvatten hebben waarmee iemand uit het water getild kan worden}{Binnen 7 seconden op je rug draaien}

\question{2}{Wat is een gedragsregel op het water?}
\answerTextFour{Houd de schippersgroet in ere}{Kom niet op andermans schip zonder toestemming}{Houd je schip en omgeving schoon}{A, B en C zijn alle drie juist}

\question{3}{Wanneer de wind plots snel draait kan dit wijzen op:}
\answerTextFour{Een weersomslag}{Opkomende bewolking}{Groter wordende golven}{A, B en C zijn alle drie juist}

\question{4}{Waarom moet je bij je boot blijven als die is omgeslagen}
\answerTextFour{Omdat naar de kant zwemmen gevaarlijk is}{Omdat je de boot niet alleen mag laten}{Om je de boot anders kwijt kunt raken}{Omdat dat gezelliger is}

\question{5}{Waar moet je op letten als je een groot schip ziet verlijeren?}
\answerTextFour{Dat deze minder goed kan sturen}{Dat deze meer ruimte in beslag neemt}{Dat de boot plots kan opschuiven}{Dat er meer zuiging is}

\question{6}{Waar is de zuiging het ergst bij een groot schip?}
\answerTextFour{De voorkant}{De voor- en achterkant}{De zijkant}{De zij- en achterkant}

\question{7}{Wat gebeurt er wanneer de wind krimpt?}
\answerTextFour{Het gaat zachter waaien}{De wind draait tegen de richting van de klok in}{De wind draait met de richting van de klok mee}{De windvlagen worden zachter}

\question{8}{De dode hoek van een groot schip is}
\answerTextFour{Waar de schipper niets kan zien}{De hoek waar de meeste zuiging is}{De hoek die het schip met het water maakt}{De achterkant van het schip}

\include{Hoofdstukken/Oefenvragen/oefenvragen_3}
\chapter{Reglementen \& Voorrangsregels}
\vspace{-120px}
\section*{Wie heeft er voorrang? Denk ook goed na waarom en vul de letter in!}
\begin{table}[h!]
\centering
\begin{tabular}{l|l|l|l|l|l|l|l|l|l|l|l}
\textbf{1} & \textbf{2} & \textbf{3} & \textbf{4} & \textbf{5} & \textbf{6} & \textbf{7} & \textbf{8} & \textbf{9} & \textbf{10} & \textbf{11} & \textbf{12} \\ \hline
 \hspace{0.5 cm} & \hspace{0.5 cm}  & \hspace{0.5 cm} & \hspace{0.5 cm} & \hspace{0.5 cm} & \hspace{0.5 cm} & \hspace{0.5 cm} & \hspace{0.5 cm} & \hspace{0.5 cm} & \hspace{0.5 cm} & \hspace{0.5 cm} & \hspace{0.5 cm}
\end{tabular}
\end{table}
\begin{figure}[h!]
    \centering
    \includegraphics[width=\textwidth]{Hoofdstukken/Oefenvragen/pdf/regelementen_1.pdf}
\end{figure}

\newpage
\begin{table}[h!]
	\centering
	\begin{tabular}{l|l|l|l|l|l|l|l|l|l|l}
		\textbf{13} & \textbf{14} & \textbf{15} & \textbf{16} & \textbf{17} & \textbf{18} & \textbf{19} & \textbf{20} & \textbf{21} & \textbf{22} & \textbf{23} \\ \hline
		\hspace{0.5 cm} & \hspace{0.5 cm}  & \hspace{0.5 cm} & \hspace{0.5 cm} & \hspace{0.5 cm} & \hspace{0.5 cm} & \hspace{0.5 cm} & \hspace{0.5 cm} & \hspace{0.5 cm} & \hspace{0.5 cm} & \hspace{0.5 cm}
	\end{tabular}
\end{table}
\begin{figure}[h!]
    \centering
    \includegraphics[width=\textwidth]{Hoofdstukken/Oefenvragen/pdf/regelementen_2.pdf}
\end{figure}
\newpage

\question{24}{Wat is de betekenis van het bord in het figuur rechts?}
\vspace*{-0.5cm}
\answerTextPicture{In-, uit-, of doorvaren verboden}{Einde van een verbod of gebod}{Verboden geluidsseinen te maken}{Verplichting voor het bord stil te houden}{Hoofdstukken/Reglementen/pdf/B5.pdf}

\question{25}{Wat betekent het volgende geluidssein: \slong \sspace  \sshort \sspace  \slong }
\answerTextFour{Attentie}{Blijf weg sein}{Ik ga bakboord uit}{Verzoek tot bediening van brug of sluis}

\question{26}{Wat is de minimale leeftijd voor het varen op een groot schip?}
\answerTextFour{Geen minimale leeftijd}{12 jaar}{16 jaar}{18 jaar}

\question{27}{Geldt het BPR op alle Nederlandse binnenwateren?}
\answerTextFour{Ja}{Ja, met uitzondering van het IJsselmeer}{Ja, met uitzondering van de Rijn }{Nee}

\question{28}{Welke schepen zijn toegestaan volgens dit bord?}
\vspace*{-0.5cm}
\answerTextPicture{Snelle motorschepen}{Kleine schepen}{Recreatievaart}{Roei- en zeilschepen}{Hoofdstukken/Reglementen/pdf/E16.pdf}

\question{29}{Welke snelheid mag hier maximaal gevaren worden?}
\vspace*{-0.5cm}
\answerTextPicture{6 kilometer per uur}{6 mijl per uur}{6 zeemijl per uur}{6 knopen}{Hoofdstukken/Reglementen/pdf/B6.pdf}

\question{30}{Mag iedereen in een snelle motorboot varen (>20km/h)?}
\answerTextFour{Ja, hier zijn geen vereisten voor}{Nee, je moet minimaal 16 jaar zijn}{Nee, je moet minimaal 18 jaar zijn}{Nee, je moet minimaal 18 jaar zijn en je KVB I of II hebben}

\include{Hoofdstukken/Oefenvragen/oefenvragen_5}
\include{Hoofdstukken/Oefenvragen/oefenvragen_6}
\chapter{Antwoorden}
\vspace{-120px}
%%%% Hoofdstuk 1 %%%%%
\begin{table}[h]
	\centering
	\begin{tabular}{c|c|c|m{9.5cm}}
	\textbf{Hfd.}       & \textbf{Vraag} & \textbf{Antwoord} & \textbf{Toelichting}                                             \\ \hline
	\multirow{10}{*}{\sffamily\bfseries{\textcolor{ocre}{\LARGE1}} } & 1  & C & Dolboord - Boeisel - Berghout - Kim - Vlak \\ \cline{2-4}          
	& 2 & B &  \\ \cline{2-4} 
	& 3 & D &  \\ \cline{2-4} 
	& 4 & A &  \\ \cline{2-4} 
	& 5 & D &  \\ \cline{2-4} 
	& 6 & A & I: Hogerwal, II: Hoge kant, III: lage kant en IV: lagerwal \\ \cline{2-4} 
	& 7 & A & Met vallen hijs je zeilen, met schoten trek je ze aan \\ \cline{2-4} 
	& 8 & C & A Aan de wind, B voor de wind, C ruime wind, D halve wind \\ \cline{2-4} 
	& 9 & B &  \\ \cline{2-4} 
	& 10 & C &  \\ 
	\end{tabular}
\end{table}

%%%% Hoofdstuk 2 %%%%%

\begin{table}[h]
	\centering
	\begin{tabular}{c|c|c|m{9.5cm}}
		\textbf{Hfd.}       & \textbf{Vraag} & \textbf{Antwoord} & \textbf{Toelichting} \\ \hline 
		\multirow{8}{*}{\sffamily\bfseries{\textcolor{ocre}{\LARGE2}} } & 1   & D         & Een reddingsvest moet je binnen \textit{15} seconden op je rug draaien  \\ \cline{2-4} 
		& 2 & D &  \\ \cline{2-4} 
		& 3 & A &  \\ \cline{2-4} 
		& 4 & A &  \\ \cline{2-4} 
		& 5 & B &  \\ \cline{2-4} 
		& 6 & D &  \\ \cline{2-4} 
		& 7 & B & Krimpen is tegen de klok in, ruimen is met de klok mee  \\ \cline{2-4} 
		& 8 & A &  \\ 
	\end{tabular}
\end{table}

%%%% Hoofdstuk 3 %%%%%

\begin{table}[h]
	\centering
	\begin{tabular}{c|c|c|m{9.5cm}}
		\textbf{Hfd.}       & \textbf{Vraag} & \textbf{Antwoord} & \textbf{Toelichting} \\ \hline 
		\multirow{5}{*}{\sffamily\bfseries{\textcolor{ocre}{\LARGE3}} } & 1   & B         &  Bij A is tegenliggende vaart mogelijk. Bij C is doorvaart verboden. Bij D is doorvaart aanstonds toegestaan.  \\ \cline{2-4} 
		& 2 & A & A gaat aanstonds open, B is al open en gaat sluiten, C is open en D is geeft geen informatie over openen  \\ \cline{2-4}  
		& 3 & B &  \\ \cline{2-4} 
		& 4 & B &  \\ \cline{2-4} 
		& 5 & C &  \\
	\end{tabular}
\end{table}

%%%% Hoofdstuk 4 %%%%%

\begin{table}[h]
	\centering
	\begin{tabular}{c|c|c|m{9.5cm}}
		\textbf{Hfd.}       & \textbf{Vraag} & \textbf{Antwoord} & \textbf{Toelichting} \\ \hline 
		\multirow{16}{*}{\sffamily\bfseries{\textcolor{ocre}{\LARGE4}} } & 1 & A & Kruisende koers regel 4: Zeilboot gaat voor spierkracht gaat voor motorboot\\ \cline{2-4} 
		& 2 & A & Tegengestelde koers regel 4: Zeilen over bakboord gaan voor \\ \cline{2-4} 
		& 3 & B & Kruisende koers regel 6.2: Loef wijkt voor lij \\ \cline{2-4} 
		& 4 & A & Kruisende koers regel 6.1: Zeilen over bakboord gaan voor \\ \cline{2-4} 
		& 5 & C & Kruisende koers regel 4: Zeilboot gaat voor spierkracht gaat voor motorboot. Dus C-A-B \\ \cline{2-4} 
		& 6 & B & Kruisende koers regel 2: Grote schepen gaan voor op kleine schepen \\ \cline{2-4} 
		& 7 & A &Kruisende koers regel 2: Grote schepen gaan voor op kleine schepen \\ \cline{2-4} 
		& 8 & B & Tegengestelde koers regel 1: Stuurboordswal gaat voor \\ \cline{2-4} 
		& 9 & B & Oplopende koers regel 5: Bij oplopen, wijk de oploper. Het opgelopen schip kan indien nodig uitwijken\\ \cline{2-4} 
		& 10 & A & Goed zeemanschap: voorkom ten alle tijden een aanvaring bij gebrek aan regels \\ \cline{2-4} 
		& 11 & A & Kruisende koers regel 6.2: Loef wijkt voor lij \\ \cline{2-4} 
		& 12 & A & Bij het oversteken van een vaarwater heb je geen voorrang \\ \cline{2-4} 
		& 13 & B & Kruisende koers regel 1: Stuurboordswal gaat voor. B vaart stuurboordswal in de vaargeul\\ \cline{2-4} 
		& 14 & B & Kruisende koers regel 1: Stuurboordswal gaat voor \\ \cline{2-4} 
		& 15 & A & Kruisende koers regel 2: Grote schepen gaan voor op kleine schepen \\ 
	\end{tabular}
\end{table}


%%%% Hoofdstuk 5 %%%%%

\begin{table}[h]
	\centering
	\begin{tabular}{c|c|c|m{9.5cm}}
		\textbf{Hfd.}       & \textbf{Vraag} & \textbf{Antwoord} & \textbf{Toelichting} \\ \hline 
		\multirow{7}{*}{\sffamily\bfseries{\textcolor{ocre}{\LARGE5}} } & 1 & A & Platteknoop voor gelijke dikte, schootsteek voor ongelijke dikte \\ \cline{2-4} 
		& 2 & B &  \\ \cline{2-4} 
		& 3 & D & Met een slipsteek kun je een mastworp `borgen' \\ \cline{2-4} 
		& 4 & D & A Mastworp, B Dubbele halve steek, C Slipsteek, D Halve steek\\ \cline{2-4} 
		& 5 & D &  \\ \cline{2-4} 
		& 6 & B &  \\ 
	\end{tabular}
\end{table}

%%%% Hoofdstuk 6 %%%%%

\begin{table}[h]
	\centering
	\begin{tabular}{c|c|c|m{9.5cm}}
		\textbf{Hfd.}       & \textbf{Vraag} & \textbf{Antwoord} & \textbf{Toelichting} \\ \hline 
		\multirow{6}{*}{\sffamily\bfseries{\textcolor{ocre}{\LARGE6}} }  & 1 & A & Lij om sneller op te loeven, loef om sneller af te vallen. Met je zwaard op verlijer je. Je grootzeil helpt juist met oploeven, vieren zal het niet sneller maken.  \\ \cline{2-4} 
		& 2 & A & De wervelingen achter het zeil komt door een te strak zeil \\ \cline{2-4} 
		& 3 & B & Met de fok val je af, dus deze vier je. Met het grootzeil loef je op dus deze trek je aan.  \\ \cline{2-4} 
		& 4 & B &  \\ \cline{2-4} 
		& 5 & B & 
	\end{tabular}
\end{table}


\part{Zeilmanoeuvres}
\label{part:manoeuvre}
\include{Hoofdstukken/Manoeuvres/oploeven_afvallen}
\include{Hoofdstukken/Manoeuvres/overstag_gijp}
\include{Hoofdstukken/Manoeuvres/hogerwal}
\include{Hoofdstukken/Manoeuvres/manoverboord}
\include{Hoofdstukken/Manoeuvres/stormrondje}

%\newpage\null\thispagestyle{empty}\newpage
%\newpage\null\thispagestyle{empty}\newpage

\begingroup
\thispagestyle{empty}
\begin{tikzpicture}[remember picture,overlay]
\node[inner sep=0pt] (last_page) at (current page.center) {\includegraphics[width=\paperwidth, page = 2]{\omslag}};
\end{tikzpicture}
\vfill
\endgroup

%----------------------------------------------------------------------------------------

\end{document}
