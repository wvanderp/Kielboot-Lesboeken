\chapter{Antwoorden}
\vspace{-120px}
%%%% Hoofdstuk 1 %%%%%
\begin{table}[h]
	\centering
	\begin{tabular}{c|c|c|m{9.5cm}}
	\textbf{Hfd.}       & \textbf{Vraag} & \textbf{Antwoord} & \textbf{Toelichting}                                             \\ \hline
	\multirow{10}{*}{\sffamily\bfseries{\textcolor{ocre}{\LARGE1}} } & 1  & C & Dolboord - Boeisel - Berghout - Kim - Vlak \\ \cline{2-4}          
	& 2 & B &  \\ \cline{2-4} 
	& 3 & D &  \\ \cline{2-4} 
	& 4 & A &  \\ \cline{2-4} 
	& 5 & D &  \\ \cline{2-4} 
	& 6 & A & I: Hogerwal, II: Hoge kant, III: lage kant en IV: lagerwal \\ \cline{2-4} 
	& 7 & A & Met vallen hijs je zeilen, met schoten trek je ze aan \\ \cline{2-4} 
	& 8 & C & A Aan de wind, B voor de wind, C ruime wind, D halve wind \\ \cline{2-4} 
	& 9 & B &  \\ \cline{2-4} 
	& 10 & C &  \\ 
	\end{tabular}
\end{table}

%%%% Hoofdstuk 2 %%%%%

\begin{table}[h]
	\centering
	\begin{tabular}{c|c|c|m{9.5cm}}
		\textbf{Hfd.}       & \textbf{Vraag} & \textbf{Antwoord} & \textbf{Toelichting} \\ \hline 
		\multirow{8}{*}{\sffamily\bfseries{\textcolor{ocre}{\LARGE2}} } & 1   & D         & Een reddingsvest moet je binnen \textit{15} seconden op je rug draaien  \\ \cline{2-4} 
		& 2 & D &  \\ \cline{2-4} 
		& 3 & A &  \\ \cline{2-4} 
		& 4 & A &  \\ \cline{2-4} 
		& 5 & B &  \\ \cline{2-4} 
		& 6 & D &  \\ \cline{2-4} 
		& 7 & B & Krimpen is tegen de klok in, ruimen is met de klok mee  \\ \cline{2-4} 
		& 8 & A &  \\ 
	\end{tabular}
\end{table}

%%%% Hoofdstuk 3 %%%%%

\begin{table}[h]
	\centering
	\begin{tabular}{c|c|c|m{9.5cm}}
		\textbf{Hfd.}       & \textbf{Vraag} & \textbf{Antwoord} & \textbf{Toelichting} \\ \hline 
		\multirow{5}{*}{\sffamily\bfseries{\textcolor{ocre}{\LARGE3}} } & 1   & B         &  Bij A is tegenliggende vaart mogelijk. Bij C is doorvaart verboden. Bij D is doorvaart aanstonds toegestaan.  \\ \cline{2-4} 
		& 2 & A & A gaat aanstonds open, B is al open en gaat sluiten, C is open en D is geeft geen informatie over openen  \\ \cline{2-4}  
		& 3 & B &  \\ \cline{2-4} 
		& 4 & B &  \\ \cline{2-4} 
		& 5 & C &  \\
	\end{tabular}
\end{table}

%%%% Hoofdstuk 4 %%%%%

\begin{table}[h]
	\centering
	\begin{tabular}{c|c|c|m{9.5cm}}
		\textbf{Hfd.}       & \textbf{Vraag} & \textbf{Antwoord} & \textbf{Toelichting} \\ \hline 
		\multirow{16}{*}{\sffamily\bfseries{\textcolor{ocre}{\LARGE4}} } & 1 & A & Kruisende koers regel 4: Zeilboot gaat voor spierkracht gaat voor motorboot\\ \cline{2-4} 
		& 2 & A & Tegengestelde koers regel 4: Zeilen over bakboord gaan voor \\ \cline{2-4} 
		& 3 & B & Kruisende koers regel 6.2: Loef wijkt voor lij \\ \cline{2-4} 
		& 4 & A & Kruisende koers regel 6.1: Zeilen over bakboord gaan voor \\ \cline{2-4} 
		& 5 & C & Kruisende koers regel 4: Zeilboot gaat voor spierkracht gaat voor motorboot. Dus C-A-B \\ \cline{2-4} 
		& 6 & B & Kruisende koers regel 2: Grote schepen gaan voor op kleine schepen \\ \cline{2-4} 
		& 7 & A &Kruisende koers regel 2: Grote schepen gaan voor op kleine schepen \\ \cline{2-4} 
		& 8 & B & Tegengestelde koers regel 1: Stuurboordswal gaat voor \\ \cline{2-4} 
		& 9 & B & Oplopende koers regel 5: Bij oplopen, wijk de oploper. Het opgelopen schip kan indien nodig uitwijken\\ \cline{2-4} 
		& 10 & A & Goed zeemanschap: voorkom ten alle tijden een aanvaring bij gebrek aan regels \\ \cline{2-4} 
		& 11 & A & Kruisende koers regel 6.2: Loef wijkt voor lij \\ \cline{2-4} 
		& 12 & A & Bij het oversteken van een vaarwater heb je geen voorrang \\ \cline{2-4} 
		& 13 & B & Kruisende koers regel 1: Stuurboordswal gaat voor. B vaart stuurboordswal in de vaargeul\\ \cline{2-4} 
		& 14 & B & Kruisende koers regel 1: Stuurboordswal gaat voor \\ \cline{2-4} 
		& 15 & A & Kruisende koers regel 2: Grote schepen gaan voor op kleine schepen \\ 
	\end{tabular}
\end{table}


%%%% Hoofdstuk 5 %%%%%

\begin{table}[h]
	\centering
	\begin{tabular}{c|c|c|m{9.5cm}}
		\textbf{Hfd.}       & \textbf{Vraag} & \textbf{Antwoord} & \textbf{Toelichting} \\ \hline 
		\multirow{7}{*}{\sffamily\bfseries{\textcolor{ocre}{\LARGE5}} } & 1 & A & Platteknoop voor gelijke dikte, schootsteek voor ongelijke dikte \\ \cline{2-4} 
		& 2 & B &  \\ \cline{2-4} 
		& 3 & D & Met een slipsteek kun je een mastworp `borgen' \\ \cline{2-4} 
		& 4 & D & A Mastworp, B Dubbele halve steek, C Slipsteek, D Halve steek\\ \cline{2-4} 
		& 5 & D &  \\ \cline{2-4} 
		& 6 & B &  \\ 
	\end{tabular}
\end{table}

%%%% Hoofdstuk 6 %%%%%

\begin{table}[h]
	\centering
	\begin{tabular}{c|c|c|m{9.5cm}}
		\textbf{Hfd.}       & \textbf{Vraag} & \textbf{Antwoord} & \textbf{Toelichting} \\ \hline 
		\multirow{6}{*}{\sffamily\bfseries{\textcolor{ocre}{\LARGE6}} }  & 1 & A & Lij om sneller op te loeven, loef om sneller af te vallen. Met je zwaard op verlijer je. Je grootzeil helpt juist met oploeven, vieren zal het niet sneller maken.  \\ \cline{2-4} 
		& 2 & A & De wervelingen achter het zeil komt door een te strak zeil \\ \cline{2-4} 
		& 3 & B & Met de fok val je af, dus deze vier je. Met het grootzeil loef je op dus deze trek je aan.  \\ \cline{2-4} 
		& 4 & B &  \\ \cline{2-4} 
		& 5 & B & 
	\end{tabular}
\end{table}

